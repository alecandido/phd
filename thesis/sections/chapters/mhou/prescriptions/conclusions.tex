% !TeX root = ../../../../main.tex

Two classes of possible prescriptions consistent with the imposed requirements
have been identified:

\begin{enumerate}
    \item full space, with some zero coefficients
    \item sliced space, with factorization scale dependent normalizations
\end{enumerate}

No one of the two is strictly allowed by eqs.\ (4.1) and (4.2) of
\cite{NNPDF:2019ubu}, since both require the usage of non-trivial
normalizations $c_i(\vec{\kappa})$, while the only normalization allowed by
eq.\ (4.2) of the paper is a global one for the whole matrix.

Moreover, eqs.\ (4.1) and (4.2) of the paper themselves does not coincide with
the correct and general \cref{eq:mhou/prescr/shifts,eq:mhou/prescr/thcovmat} in
this thesis, because eq.\ (4.2) in the paper is already defined at the level of
a subspace $V_m$ of the large space $\mathcal{V}$, and while this is described
in the following, no proof is given of the compatibility of these subspaces as
projections of the full one.

Finally, the notation used in \cite{NNPDF:2019ubu} is confusing, since eq.\
(4.2) of the paper gives the impression that the first shift $\Delta_i$ and the
second shift $\Delta_j$ are potentially evaluated on different points, while
the point has to be always the same, simply the actual dependence of the shifts
is on two different scales.
We advocate for a more explicit and transparent syntax, at least while defining
the general landscape for prescriptions (while at the individual prescription
level a more concise syntax might even be useful, if properly introduced in
relation to the general one).
