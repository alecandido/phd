% !TeX root = ../../../main.tex

The investigation in \cite{NNPDF:2019ubu} stopped at \nlo, because of various
technical limitations, and the lack of a proper benchmark to assess the
reliability of available implementations.
%
An \nnlo fit based on the theory covariance matrix formalism is still missing,
and this specific target will be achieved with the tools offered by the new
theory pineline (cf. \cref{ch:pine}).

A first update regards the actual implementation of scheme B, introduced in
\cref{sec:mhou/pdf}.
Indeed, scale variations in scheme A-like fashion has been the first and
default way they appeared in the evolution programs, starting with programs
like \pegasus.
In order to apply them, the \pdf anomalous dimensions have to be evaluated at a
scale shifted from the usual one, and a few extra terms appears.
%
But, being implemented at the anomalous dimensions level, these contributions
are resummed by the solution of the differential equations, and essentially
exponentiated (the solution of a linear differential equation is a path ordered
exponential of the associated kernel).
%
But this is not the contribution required by scheme B, and violates the scheme
C equivalence.
To get the expanded result, an extra piece has to be multiplied to the evolved
\pdf, so after solving the differential equation.
%
For this reason, \eko ships both the kind of factorization scale variations
(the only scale variations on the evolution side), dubbed
\textit{exponentiated} and \textit{expanded}, and the user can choose an actual
solution conformal to the scheme B prescriptions.
%
Since \eko does not produce evolved \pdfs, but evolution operators, the extra
piece appearing in the expanded case is also implemented at the anomalous
dimensions level, Mellin inverted, and included as an extra factor in the final
operator returned as output (i.e.\ it is not returned individually).

Another delicate point is the role of the strong coupling $\alpha_s$ in the
computation of factorization scale variations.
Indeed $\alpha_s$ appears in two places: as the parameter of the perturbative
series, evaluated at the scale of the process, and in the solution of \dglap
equations.
%
Indeed, the running coupling $\alpha_s(Q^2)$ is a monotonically decreasing
function of the scale, and contains the only scale dependence in the anomalous
dimensions.
Therefore, it is convenient to change variable, and solve \dglap as function of
$\alpha_s$, instead of the factorization scale.
%
This usage might suggest a relation between the strong coupling and the
factorization scale, and then the value used should be affected by the
related variations.
Instead, this is only used as a monotonic function, so for its mathematical
properties and the role in the equation (stemming from anomalous dimensions
perturbative expansion), and there is no implication about a physical relation.
%
Essentially, $\alpha_s$ value is only sensitive to renormalization scale
variations, that does not affect evolution at all, but it is used there as a
function of factorization scale.
This usage led to a clash in the options used, and a slightly wrong result when
both variations were applied, and it has been fixed in the current pineline.

Moreover, scheme C can also be obtained without requiring any information about
scale variations from the generators used.
\dots

Finally, using both the ingredients provided by the \eko and \yadism libraries,
we benchmarked our implementations of scale variations with some analytical
benchmarks, based on the expressions for some specific \dis contributions, at
order by order and for specific partonic channels.
\dots
