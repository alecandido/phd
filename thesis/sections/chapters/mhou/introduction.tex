% !TeX root = ../../../main.tex

The first fundamental application of the integrated pineline (cf.\
\cref{ch:pine}) will actually the inclusion of \acrfull{mhou} at \nnlo in the
\nnpdfr{4.0} fit.

\pdfs are non-perturbative objects, so it may seem counter-intuitive that their
accuracy depends on perturbative series truncation.
%
This is a direct consequence of extracting them from high energy collisions
data: they are completely determined by physics that happens in the
perturbative regime, and the map discussed at the beginning of \cref{ch:pine}
(the one that connects data to \pdfs) is completely determined by \pqft
calculations.
%
So, the origin of the perturbative order of \pdf sets is exactly determined by
the theory predictions used during the extraction: a \nnlo set is a \pdf set
that has been fitted using theory predictions at \nnlo.
%
A \pdf set directly computed with non-perturbative methods would have no
perturbative order associated, even when used in a perturbative calculation
\footnote{
	From that point of view would be an \textit{all-order} object, even though
	it might be subject to other kinds of approximations.
}\footnote{
	Also consider that \dglap evolution is perturbative, so, once evolved, it
	acquires again a dependency on the perturbative truncation.
}.

The perturbative series enters in the \pdf in two different places: the
partonic cross section calculations (those encoded in \textit{grids}) and the
\dglap evolution\footnote{
	That technically is used twice: during the fit, to bridge data with the
	boundary condition candidate, and to evolve the final boundary condition to
	all scales.
	But considering the \pdf a function of two variables ($z$ and $\mu_F^2$)
	consistently, the abstract evolution flow used is a single one.
}.
%
In principle, these are two different perturbative orders, thus there is not a
single truncation, but two of them, and they can happen at two different
orders.
%
Still, the two objects are not completely decoupled: \dglap evolution arise
from collinear divergences, subtracted by the chosen factorization scheme.
These collinear logarithms appear as well in the partonic cross sections, so it
is important to properly account for them, avoiding double counting.
%
The whole picture of collinear subtractions is deeply connected to treatment of
quark masses, better discussed in \cref{ch:dis}, since a finite value of the
mass regulates the collinear divergence on its own.
%
Therefore, the double perturbative order already appears in the partonic cross
sections calculations, where the \fns chosen can account for light and heavy
quarks at two distinct orders (cf.\ \cite{Forte:2010ta}, in particular the
FONLL-B scheme).
