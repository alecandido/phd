\section{Parton distributions from the parton model to QCD}
\label{sec:intro}

In the naive parton model parton distribution functions (PDFs) are probability
densities for a parton to carry a given fraction of their parent hadron's
momentum. This simple picture remains true at leading order (LO) in QCD where
each PDF is proportional to a physically measurable (in principle) cross
section: it can consequently still be viewed as a probability and it is
non-negative. We will henceforth refer to the property of being non-negative as
``positivity'' (while really meaning ``positive-semidefiniteness'').

As is well known~\cite{Altarelli:1998gn}, positivity of the PDFs is in general
violated at higher perturbative orders, where physical cross sections are the
convolution of the PDFs with a partonic cross section.
Partonic cross sections are beset by collinear singularities, whose subtraction
and factorization into the PDF spoils in general the separate positivity of the
subtracted partonic cross sections and of the PDFs.
Therefore, the positivity condition on PDFs beyond LO becomes a positivity
constraint of physical observables~\cite{Altarelli:1998gn,Forte:1998kd}. 
When determining
PDFs from data, this constraint must be imposed on physical observables, rather
than PDFs themselves: for instance by using Lagrange multipliers, or pseudodata
for suitable positivity observables, such as for example hypothetical
deep-inelastic structure functions corresponding to gauge bosons that couple to
only one quark flavor~\cite{Ball:2008by}.


These positivity constraint 
may have a significant impact on
PDF determination, especially in regions where there are little or no direct
constraints coming from experimental data. For example, in a recent
study of the strange PDF~\cite{Faura:2020oom} it was found that
adding to the dataset a positivity constraint for the charm structure
function $F_2^c$ significantly impacts the flavor decomposition of the
quark sea in the
large $x$ region. However, adding positivity constraints in this way,
as constraints on pseudo-observables, is cumbersome from a methodological point
of view, and theoretically not satisfactory. Indeed, as  the target
accuracy of the PDFs increases, and PDFs are used more and more
for new physics studies and searches in regions in which the PDFs are
experimentally 
unconstrained, it becomes necessary to enforce 
an increasingly
elaborate set of positivity
constraints~\cite{Ball:2010de,Ball:2014uwa} for
a set of suitably chosen and tuned pseudo-observables. This poses
obvious problems of fine-tuning. In fact,
universality of PDFs suggests that  positivity constraints
should be imposed in a process-independent way, without having to rely
on a specific choice  of processes, and therefore, that it ought to be
possible to impose the constraint at the level of PDFs.

Here, we address this issue head-on by constructing a subtraction scheme in
which PDFs are positive, and which we
refer to as a ``positive'' factorization scheme. We do this by
studying the way collinear subtraction is performed in the \msbar{}
scheme, and showing that negative partonic
cross sections arise as a consequence of
over-subtraction of a positive contribution\footnote{A subtlety is related to the fact that
  generally partonic cross sections are the sum of an ordinary
  function of the scaling variable, and a distribution localized at
  the kinematic threshold of the scaling variable. Here, by ``negative
  cross section'' we mean that the function (i.e., non-distributional
  part of the cross section) is negative. For positivity to hold, the
  distributional part must also be positive in the sense that it gives
a positive result when integrated over a positive test function. As we
shall see below, this condition turns out to be automatically satisfied
in \msbar{} and related schemes.}.
This is chiefly due to the fact that the subtraction is
performed at a scale which, as the  kinematic threshold for production
of the final state is approached,
is higher than the actual physical scale;
and also, in gluonic
channels, due to the way
the $d$-dimensional averaging over gluon polarizations is treated in
dimensional regularization.

Once these effects are taken into account
it is possible to formulate a subtraction prescription such that
partonic cross sections remain
positive. Effectively, this choice of subtraction corresponds to a ``physical''
scheme, in which the scale choice is directly related to the scale of
parton radiation.
Of course, the
positive factorization scheme is not unique, since any
further scheme change through a matrix with positive entries
(``positive matrix'', henceforth) would leave 
the partonic cross sections positive.  It is then possible to show that there exist
schemes in which PDFs also remain positive, so that the positive
hadronic cross section is obtained by convoluting positive
partonic cross sections with positive PDFs.

The availability of positive schemes can be
advantageous in the 
context of PDF determination. Indeed, if PDFs are
parametrized in the positive scheme, positivity can be enforced by
choice of parametrization. Results in the commonly used \msbar{} scheme
can then be obtained by scheme transformation.

However, perhaps surprisingly, this turns out not to be necessary:
indeed, using the explicit form of the scheme change
matrix from the positive scheme to \msbar{} it is easy to prove that in
the perturbative region  PDFs remain positive in
\msbar{}. Hence, the common lore that \msbar{} PDFs might be negative
beyond LO
turns out to be incorrect. Positivity of the fitted PDFs can then be imposed
using the standard methodology in the
\msbar{} scheme.

The paper is organized as follows. In \cref{sec:subtr} we show
how negative partonic cross sections arise due to over-subtraction. We
start with the   prototypical case of deep-inelastic scattering (DIS): we
review the computation of coefficient functions at next-to-leading
order (NLO); we show how over-subtraction arises in the \msbar{} scheme
and how it can be fixed by choosing a suitable subtraction
prescription; we then discuss how this works in the general case of
hadronic processes, where we can define a ``positive'' subtraction prescription
which preserves
positivity of all partonic cross sections.
In \cref{sec:scheme} we then turn to
positivity of PDFs: first, we use our positive subtraction
prescription to define a positive factorization scheme; then we show
how positivity of PDFs is preserved in this factorization scheme; and
finally by studying the transformation from the positive scheme to  \msbar{}
we prove that positivity is preserved in \msbar{} in the perturbative
regime. The bulk of our discussion will be at NLO, and its validity
beyond NLO will be addressed in the end of \cref{sec:scheme}.

