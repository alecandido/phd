% !TeX root = ../../../main.tex

The origin of this study has been the observation, brought to \nnpdf by
external users, that the adoption of \nnpdfr{3.1} in some searches produced
negative results for the central values of physical observables.

The problem was identified in the \pdf set having negative values for some
flavors, specifically in the large-$x$ region, which is probed by searches, but
not covered by data.
%
However, already \nnpdfr{3.1} imposed the positivity of a set of physical
observables, that is a relevant constraint on the \pdf shape.
But it is not possible to cover, with a finite amount of observable values, the
whole spectrum of possible measurements, thus they guarantee more the
positivity of standard data-region values (e.g.\ \dy distributions, or \dis
structure functions), but not of any possible \bsm search.
%
Moreover, an too strong assumptions in the extrapolation region might also
generate some bias towards the \sm itself, and the inclusion of constraints
on \bsm observables would be at the very least arbitrary.

Yet, it is true that in the known cases observables negativity can be traced
back to \pdfs negativity.
%
Indeed, \lo observables are positive by construction (being squared
amplitudes), and convoluted with a positive \pdf would yield a positive result,
though it is not generally true for all orders, since they are affected by
factorization subtractions, that can generate negative results also from
positive \pdfs.

For this reason, we wondered if it were possible to obtain a scheme that would
guarantee the positivity of the \pdfs, in such a way that two results would be
achieved at the same time: \lo observables would be positive with these \pdfs,
and a further theoretical constraint could be imposed on the fit, augmenting
the physical information embedded in the fit.

As explained later on, the existence goal is easily obtained, since it is
possible to construct \enquote{physical} schemes in which the \pdf is anchored
to an observable, and thus positive by construction.
%
Unfortunately, using \pdfs in this schemes would be rather unpractical, since
dedicated calculations would be required, adding the details (coefficient
functions) of the process linked to the \pdfs to any other process as well.
So the further requirement we imposed was to obtain a positive subtraction
scheme as close as possible from \msbar, such that \pdfs could be fitted in
that scheme, and tree level observables would become positive anyhow, since
they do not require a specific subtraction (thus the difference with respect to
\msbar would be higher order).
%
We started from the assumption that \msbar itself were not a negative scheme,
and we tried to enhance positivity working on the negativity of the coefficient
functions in $N$-space.
%
We found out in the first place that $N$-space positivity does not coincide
with $x$-space one, and the relation is rather non-trivial.
We wanted the \pdf values to be positive, not its moments, so we needed to work
on n $x$-space transformation, even though this involved necessarily
distributions.
%
Once we had a reasonable candidate, obtained tracing back the structure of
\msbar subtraction in $d$-dimensions, we started proving its positivity, and
studying the relation with the \msbar.
There, it strangely appeared from the explicit change of scheme that \msbar
\pdfs were more positive than those in the new scheme.
%
Finally, after convincing ourselves of this fact, we turned our argument to
prove the positivity of \msbar scheme itself.
This is the most convenient result for the \pdf fit, achieving the goal of
healing the original \textit{issue} with \nnpdfr{3.1}, but it is in no way
assuring the positivity of resulting observables, that remain beset by
subtraction and perturbative truncation.
