In this case as well, for better comparison with \cite{NNPDF:2019ubu}, we
introduce the factor of $s$ in the normalization of \cref{eq:sliced-norm}, thus
\begin{equation}
    c_i(\vec{\kappa})^2 = \frac{s_m}{m \cdot \abs{v(\kappa_F)}^{p-1}}
\end{equation}

Furthermore, same as in \cref{sec:main-ex} (on purpose, to stress comparison)
we consider the case of only two data points (1 and 2) belonging to two
distinct processes.
With this limited case it is harder to appreciate the difference in the
constructions of \cref{sec:deriv} and \cref{sec:slices}, since it actually lies
in the way the different three dimensional shapes for pair of processes are
reconciled in the full $p+1$-dimensional space.
However, this difference has already been stressed in the abstract construction
of the two classes of prescriptions, thus the purpose of this examples is
different: to showcase the different expressions obtained fully explicitly.
For this aim the choice of considering just two points is fully satisfactory.

For this second set of examples there is no need to rewrite the full set of
terms: they are the exact same of \cref{sec:main-ex}, the only difference will
be in the coefficients, that now might depend on the value of $\kappa_F$
because of the space structure (and they will always depend on it).

Thus, the expressions for the \textit{diagonal} and \textit{off-diagonal} cases
with only two process, $p = 2$, in this second class of prescriptions are the
following:
\begin{description}
    \item[diagonal] effectively two-dimensional, since both the shifts depend only on two scales
        \begin{align}
            S_{11} &= \sum_{\vec{\kappa} \in \mathcal{V}} \Delta_1(\vec{\kappa}) \Delta_1(\vec{\kappa}) =\\
                   &= \sum_{\kappa_F \in v_F} \frac{s_m}{\abs{v(\kappa_F)} \cdot m} \sum_{\vec{\kappa}_R \in \mathcal{V}(\kappa_F)}\delta_1(\vec{\kappa})^2 =\\
                   &= \frac{s_m}{m} \sum_{\kappa_F \in v_F} \sum_{\kappa_{R,1} \in v(\kappa_F) } \delta_1(\kappa_F, \kappa_{R, 1}, 0)^2
            \label{eq:slices-explicit-diag}.
        \end{align}
        where in the last step a single value has been chosen for $\kappa_{R,
        2}$, since $\delta_1$ does not depend on this scale (this trivial sum
        cancels with the factor of $\abs{v(\kappa_F)}$ in the denominator).

        Notice that the last sum $\sum_{\kappa_F \in v_F} \sum_{\kappa_{R,1}} =
        \sum_{\substack{(\kappa_F, \kappa_{R,1}) \in v_m^1 }}$, thus the
        finally formula for the diagonal case is the same of
        \cref{eq:main-explicit-diag}.
        While this is not a proof of the general case, it is simple to show (in
        essentially the same way of above) that this is the formula obtained for
        any number of processes $p$.
    \item[off-diagonal] effectively three-dimensional, that only for this
        specific problem coincide with the whole space
        \begin{align}
            S_{12} &= \sum_{\vec{\kappa} \in \mathcal{V}} \Delta_1(\vec{\kappa}) \Delta_2(\vec{\kappa}) =\\
                   &= \sum_{\kappa_F \in v_F} \frac{s_m}{\abs{v(\kappa_F)} \cdot m} \sum_{\vec{\kappa}_R \in \mathcal{V}(\kappa_F)}\delta_1(\vec{\kappa})\delta_2(\vec{\kappa})\\
                   &= \frac{s_m}{m} \sum_{\kappa_F \in v_F} \frac{1}{\abs{v(\kappa_F)}} \sum_{\vec{\kappa}_R \in \mathcal{V}(\kappa_F)}\delta_{12}(\kappa_F, \kappa_{R,1}, \kappa_{R,2})
            \label{eq:slices-explicit-off-diag}.
        \end{align}
\end{description}

Since the space of this second class is engineered to give the same terms of
the first one (both diagonal and off-diagonal), and the normalizations are
chosen such to obtain uniform coefficients for the diagonal case (and then they
are the exact same of the first class, as noted above), the only difference
will be in the \textbf{coefficients of the off-diagonal} case, and they can
only depend on the factorization scale $\kappa_F$.
For this reason, we will not repeat the full construction of the previous
section, but just adopt a concise notation to make the different coefficients
explicit in the off-diagonal expressions:
\begin{equation}
    S_{12} = \frac{s_m}{m \cdot k_m} \left(c_m(-) \deltagroup{-} +
        c_m(0) \deltagroup{0} + c_m(+) \deltagroup{+}\right)
\end{equation}
where:
\begin{itemize}
    \item $k_m$ is the least common multiple of the $\abs{v(\kappa_F)}$, in
        order to leave integer coefficients in the sum
    \item $c_m(\kappa_F)$ is the leftover the $1/\abs{v(\kappa_F)}$ once
        $1/k_m$ has been factored out
    \item $\deltagroup{\kappa_F}$ is a placeholder for all the terms with that
        value of $\kappa_F$, as they have been spelled out in the corresponding
        prescription in \cref{sec:main-ex}
\end{itemize}

\subsubsection{9 points}

The specification of this prescription is almost the same of the corresponding
one for the first class:
\begin{align}
    \label{eq:5specs-slices}
    v_9(-) &= v_9(+) = \{-, 0, +\}\\
    v_9(0) &= \{-, +\}\\
\end{align}
Therefore, the resulting \textit{off-diagonal} expression is:
\begin{align}
    S_{12} &= \frac{2}{8 \cdot 6} \left(2\ \deltagroup{+} + 2\ \deltagroup{-} + 3\ \deltagroup{0}\right)\\
        &= \frac{1}{24} \left(2\ \deltagroup{+} + 2\ \deltagroup{-} + 3\ \deltagroup{0}\right)
\end{align}

\subsubsection{5 points}

For this prescription, the difference is a bit more relevant, mainly in terms
of the overall factor, since no one of the $v_5(\kappa_F)$ spaces has the
maximal allowed cardinality, i.e. $3$\footnote{
    Of course even 3 is completely arbitrary, as explained in
    \cref{eq:1dim-space}, and the related note, but both classes of
    prescriptions are perfectly \textit{adaptive} w.r.t.\ this value,
    i.e.\ their definitions work perfectly fine in the general case.
}
\begin{align}
    \label{eq:9specs-slices}
    v_5(-) &= v_9(+) = \{0\}\\
    v_5(0) &= \{-, +\}\\
\end{align}
Therefore, the resulting \textit{off-diagonal} expression is:
\begin{align}
    S_{12} &= \frac{2}{4 \cdot 2} \left(2\ \deltagroup{+} + 2\ \deltagroup{-} + \deltagroup{0}\right)\\
        &= \frac{1}{4} \left(2\ \deltagroup{+} + 2\ \deltagroup{-} + \deltagroup{0}\right)
\end{align}

\subsubsection{$\bar{5}$ points}

It is worth to analyze separately also this prescription: the former two are
enough to exemplify the regular cases, but this one is slightly
\textit{degenerate}.
Indeed, one of the spaces is actually empty.
\begin{align}
    \label{eq:5bar-specs-slices}
    v_5(-) &= v_9(+) = \{-, +\}\\
    v_5(0) &= \{\}\\
\end{align}
We need to generalize a bit the definition given above: $k_m$ is chosen to be
the least common multiple of all \textbf{non-zero} coefficients.
Finally, the \textit{off-diagonal} expression for this prescription is:
\begin{align}
    S_{12} &= \frac{2}{4 \cdot 2} \left(\deltagroup{+} + \deltagroup{-}\right)\\
        &= \frac{1}{4} \left(\deltagroup{+} + \deltagroup{-}\right)
\end{align}
