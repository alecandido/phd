% !TeX root = ../../../main.tex

The fraction of the proton momentum carried by charm quarks is given by
\begin{equation}
\label{eq:ic/charm_momentum_fraction}
\left[ c\right] = \int_0^1dx\, x c^+(x,Q^2) \, .
\end{equation}
Model predictions, as mentioned, are typically provided up to an overall
normalization, which in turn determines the charm momentum fraction.
%
Consequently, the momentum fraction is often cited as a characteristic
parameter of intrinsic charm.
%
It should however be borne in mind that, even in the absence of intrinsic
charm, this charm momentum fraction is nonzero due to the perturbative
contribution.

In \cref{tab:ic/momfrac_lowQ} we indicate the values of the charm momentum
fraction in the 3\fns for our default charm determination and in the 4\fns  (at
$Q=1.65$ GeV) both for the default result and for perturbative charm \pdf (see
\cref{sec:ic/consistency}).
%
We provide results for  three different values of the charm mass $m_c$ and
indicate separately the \pdf and the MHO uncertainties.
%
The 3\fns result is scale-independent, it corresponds to the momentum fraction
carried by intrinsic charm, and it vanishes identically by assumption in the
perturbative charm case.
%
The 4\fns result corresponds to the scale-dependent momentum fraction that
combines the intrinsic and perturbative contribution, while of course it
contains only the perturbative contribution in the case of perturbative charm.
%
As discussed in \cref{sec:ic/consistency}, the large uncertainty associated to
higher order corrections to the matching conditions affects the 3\fns result
(intrinsic charm) in the default case, in which the charm \pdf is determined
from data in the 4\fns scheme, while it affects the 4\fns result for
perturbative charm, that is determined assuming the vanishing of the 3\fns
result.

For our default determination, the charm momentum fraction in the 4\fns at low
scale differs from zero at the $3\sigma$ level.
%
However, it is not possible to tell whether this is of perturbative or
intrinsic origin, because, due to  the large \mhou in the matching condition,
the intrinsic (3\fns) charm momentum fraction is compatible with zero. This
large uncertainty is entirely due to the small $x\lsim 0.2$ region, see see
\cref{fig:ic/charm_content_3fns}~(right).
%
Accordingly, for perturbative charm the low-scale 4\fns momentum fraction is
compatible with zero.
%
Consistently with the results of \cref{sec:ic/charm_stability_4fns}, the 4\fns
result is essentially independent of the value of the charm mass, but it
becomes strongly dependent on it if one assumes perturbative charm.

%%%%%%%%%%%%%%%%%%%%%%%%%%%%%%%%%%%%%%%%%%%%%%%%%%%%%%%%%%%%%%%%%%%%%%%%%%%%%%%
\begin{table}[t]
  \footnotesize
  \centering
    \renewcommand{\arraystretch}{1.30}
    \begin{tabularx}{\textwidth}{C{1.5cm}ccc>{\centering\arraybackslash}X}
      \toprule
      Scheme  & $Q$ & Charm \pdf & $m_c$  &  $\left[ c\right]~\left(\%\right)$ \\
      \midrule
      \midrule
    3\fns  & --  &default  &  1.51 GeV  &   $ 0.62\pm0.28_\textrm{ pdf}\pm 0.54_\textrm{ mhou} $ \\
    3\fns  & --  &default  &  1.38 GeV  &   $ 0.47\pm0.27_\textrm{ pdf}\pm 0.62_\textrm{ mhou} $ \\
    3\fns  & --  &default  &  1.64 GeV  &    $ 0.77\pm0.28_\textrm{
      pdf}\pm 0.48_\textrm{ mhou} $ \\
      \midrule

    4\fns  & 1.65 GeV  & default  &  1.51 GeV  &   $0.87 \pm 0.23_\textrm{ pdf}$  \\
    4\fns  & 1.65 GeV  & default &  1.38 GeV  &   $0.94 \pm 0.22_\textrm{ pdf}$  \\
    4\fns  & 1.65 GeV  & default   &  1.64 GeV  &  $0.84 \pm 0.24_\textrm{ pdf}$  \\
      \midrule
    \midrule
    4\fns  & 1.65 GeV   & perturbative  &  1.51 GeV  &   $0.346\pm 0.005_\textrm{ pdf}\pm 0.44_\textrm{ mhou}$ \\
    4\fns  & 1.65 GeV   & perturbative  &  1.38 GeV  &    $0.536\pm 0.006_\textrm{ pdf}\pm 0.49_\textrm{ mhou}$ \\
    4\fns  & 1.65 GeV   & perturbative  &  1.64 GeV  &    $0.172\pm 0.003_\textrm{ pdf}\pm 0.41_\textrm{ mhou}$ \\
    \bottomrule
    \end{tabularx}
\vspace{0.3cm}
\caption{\label{tab:ic/momfrac_lowQ}
  The charm momentum fraction, \cref{eq:ic/charm_momentum_fraction}.
  %
  We show  results both in the 3\fns and the 4\fns (at $Q=1.65$ GeV)
  for our default charm, and also in the 4\fns for perturbative charm.
  %
We provide results for  three different values of the charm mass $m_c$ and
indicate separately the \pdf and the MHO uncertainties.
}
\end{table}
%%%%%%%%%%%%%%%%%%%%%%%%%%%%%%%%%%%%%%%%%%%%%%%%%%%%%%%%%%%%%%%%%%%%%%%%%%%%%%%

%%%%%%%%%%%%%%%%%%%%%%%%%%%%%%%%%%%%%%%%%%%%%%%%%%%%%%%%%%%%%%%%%%%
\begin{figure}[h]
  \begin{center}
     \includegraphics[width=0.60\linewidth]{ch-ic/charm_momfrac_qdep.pdf}
    \caption{\small 
      The 4\fns charm momentum fraction in \nnpdfr{4.0} as a function of scale $Q$,
      both for the default and perturbative charm cases,
      for a charm mass value of $m_c=1.51$ GeV.
      %
     The inset zooms on the low-$Q$ region and includes the 3\fns
     (default) result
     from \cref{tab:ic/momfrac_lowQ}. 
     %
     Note that the uncertainty includes the \mhou for the 3\fns default
     and 4\fns perturbative charm cases, while it is the \pdf
     uncertainty for the 4\fns default charm case.
  \label{fig:ic/comparison_IC_models} }
\end{center}
\end{figure}
%%%%%%%%%%%%%%%%%%%%%%%%%%%%%%%%%%%%%%%%%%%%%%%%%%%%%%%%%%%%%%%%%%%%%%

The 4\fns charm momentum fraction is plotted as a function of scale
in \cref{fig:ic/comparison_IC_models}, both in the default case and
for perturbative charm, with the 3\fns values and the detail of the low-$Q$ 
4\fns results shown in an inset.
%
The dependence on the value of the charm mass
is shown in \cref{fig:ic/charm_momfrac_qdep_mc}.
The large MHOUs on the 3\fns result, and on the 
4\fns result in the case of perturbative charm, are apparent.
The stability of the default result upon variation of  the value of
$m_c$, and the strong dependence of the perturbative charm result on
$m_c$, are  also clear.
Both the large \mhou uncertainty, and the strong dependence on
the value of $m_c$
for perturbative charm are seen to persist up to large scales.


%%%%%%%%%%%%%%%%%%%%%%%%%%%%%%%%%%%%%%%%%%%%%%%%%%%%%%%%%%%%%%%%%%%
\begin{figure}[t]
  \begin{center}
    \includegraphics[width=0.49\linewidth]{ch-ic/charm_momfrac_qdep_mc.pdf}
    \includegraphics[width=0.49\linewidth]{ch-ic/charm_momfrac_qdep_mc_pert.pdf}
    \caption{\small
    Same as \cref{fig:ic/comparison_IC_models} for different values
    of the charm mass. Note that the 3\fns momentum fraction for
     perturbative charm vanishes identically by assumption.
   \label{fig:ic/charm_momfrac_qdep_mc} }
\end{center}
\end{figure}
%%%%%%%%%%%%%%%%%%%%%%%%%%%%%%%%%%%%%%%%%%%%%%%%%%%%%%%%%%%%%%%%%

It is interesting to understand in detail the impact of the \mhou on
the momentum fraction carried by intrinsic charm. To this purpose, we
have computed  the truncated momentum integral, i.e.\ 
  \cref{eq:ic/charm_momentum_fraction} but only integrated down to
  some  lower
  integration limit $x_\textrm{ min}$:
  \begin{equation}
\label{eq:ic/charm_momentum_fraction_truncated}
\left[ c\right]_\textrm{ tr}(x_\textrm{ min}) \equiv \int_{x_\textrm{ min}}^1dx\, x c^+(x,Q^2) \, .
\end{equation}
Note than in the 3\fns   $x
c^+(x)$ does not depend on scale, so  this becomes
a scale-independent quantity.
%
The result for our default intrinsic charm determination is displayed
in \cref{fig:ic/charm_momfrac_xmin_dep}, as a function of
of the lower integration limit $x_\textrm{ min}$.
%
It is clear that for $x_\textrm{ min} \gtrsim 0.2$ the truncated momentum
fraction  differs significantly from zero, thereby providing evidence
for intrinsic charm with similar statistical  significance as the
local pull shown in \cref{fig:ic/Zc} bottom left.
%
For $x \lsim 0.2$
this  significance is then washed out
by the large MHOUs.

Hence, while the total momentum fraction has been traditionally adopted
as a measure of intrinsic charm, 
our analysis shows that, once MHOUs are accounted for, the information
provided by the total momentum fraction is limited, at least with
current data and theory.



%%%%%%%%%%%%%%%%%%%%%%%%%%%%%%%%%%%%%%%%%%%%%%%%%%%%%%%%%%%%%%%%%%%
\begin{figure}[h!]
  \begin{center}
    \includegraphics[width=0.65\linewidth]{ch-ic/charm_momfrac_xmin_dep_3fns.pdf}
    \caption{\small The value of the truncated charm momentum integral,
      \cref{eq:ic/charm_momentum_fraction_truncated}, as a function
      of the lower integration limit $x_\textrm{ min}$
      for our baseline determination of the 3\fns intrinsic charm \pdf.
      %
      We display separately the \pdf and the total (\pdf+\mhou) uncertainties.
  \label{fig:ic/charm_momfrac_xmin_dep} }
\end{center}
\end{figure}
%%%%%%%%%%%%%%%%%%%%%%%%%%%%%%%%%%%%%%%%%%%%%%%%%%%%%%%%%%%%%%%%%%%%%%
