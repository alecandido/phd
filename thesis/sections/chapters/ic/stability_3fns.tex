% !TeX root = ../../../main.tex

We now repeat the stability and uncertainty
study of the previous section, but for our final
result, namely the intrinsic charm \pdf. The main difference to be kept
in mind is that the uncertainty now also includes the dominant \mhou,
due to the matching condition required in order to determine the 3\fns
\pdf from the 4\fns result. In order to get a complete picture, we now
add a further set of dataset variations.

\paragraph{Dependence on the input dataset.}
%
\cref{fig:ic/charm_dataset_dep_nf3} displays the dataset variations shown in
\cref{fig:ic/charm_dataset_dep}, now for the intrinsic (3\fns) charm \pdf, but
with the total uncertainty now being the sum in quadrature of the \pdf and
\acrlong{mhou}, with the latter determined as the difference between results
obtained using \nnlo and \nnnlo matching.
%
Additionally, we also performed a few extra  dataset
variations: a fit without any $W, Z$ production data from \atlas and \cms,
a fit without jet data, a fit without $Z$ $p_T$ measurements, and a fit without
\hera structure function data.
%
Note that the collider-only dataset includes both \hera electron-proton collider
data and Tevatron and \lhc hadron collider data, but not fixed-target
\acrlong{dis} and \acrlong{dy} production data.

Results are qualitatively very similar to those seen in the 4\fns, a
consequence of the fact that we are focusing on the large-$x$ region where the
effect of the matching is moderate, though now the presence of a
valence-like peak in all determinations is even clearer, specifically
for the \dis-only fit where it was less pronounced in the 4\fns.
%
Note however that the \dis-only determination exhibits larger uncertainties
(up to factor 2) and point-by-point fluctuations, and is dominated by
relatively old fixed-target measurements.
%
Comparison of all the dataset variations shows that, in terms of their impact
on intrinsic charm, hadron collider data are generally more important that
deep-inelastic data, that among the former the \lhcb inclusive $W,Z$ data are
playing a dominant role, and that jet observables also play a non-negligible
role.

It should be stressed that the agreement between results found using \dis data
and hadron collider data is highly nontrivial, since in the region relevant for
intrinsic charm these determinations are based on disjoint datasets and are 
affected by very different theoretical and experimental uncertainties: in
particular, potential higher-twist effects in the \dis observables are highly
suppressed for collider observables.
%
this respect, a \dis-only determination of intrinsic charm
is potentially affected by sources of theory uncertainties, such as higher
twists, which are not accounted for in global \pdf determinations.

We conclude that the characteristic valence-like peak structure at large-$x$
predicted by non-perturbative intrinsic charm models
(\cref{fig:ic/charm_content_3fns} in \cref{sec:ic/intro}) is always present
even under very significant changes of the dataset.

\paragraph{Dependence on the parametrization basis.}
%
\Cref{fig:ic/charm_basisdep_3FNS} displays the comparison between the intrinsic
charm \pdf determined with the default evolution basis choice, and the flavor
basis. Complete consistency of central values is found, with somewhat larger
uncertainties in the case of the flavor basis, due to the more  challenging
fitting environment in this basis (see the discussion in \cite{Ball:2021leu}).

\paragraph{Dependence on the charm mass value.}
%
The study of the charm mass dependence is particularly interesting, because the
intrinsic component should be independent of it, hence the residual dependence
seen in \cref{fig:ic/charm_fitted_mcdep} in the 4\fns, due to the mass
dependence of the perturbative component that could not be reabsorbed in the
fitting, should no longer be present. 
Results are shown in \cref{fig:ic/mass_variations_Quad_MHOU}, and it is
apparent that indeed the dependence on the charm mass has all but disappeared.

%%%%%%%%%%%%%%%%%%%%%%%%%%%%%%%%%%%%%%%%%%%%%%%%%%%%%%%%%%%%%%%%%%%
\begin{figure}[H]
  \centering
  \includegraphics[width=0.49\linewidth]{ch-ic/EMC_Quad_MHOU.pdf}
  \includegraphics[width=0.49\linewidth]{ch-ic/DIS_only_Quad_MHOU.pdf}
  \includegraphics[width=0.49\linewidth]{ch-ic/collider_only_Quad_MHOU.pdf}
  \includegraphics[width=0.49\linewidth]{ch-ic/noLHCb_Quad_MHOU.pdf}
  \includegraphics[width=0.49\linewidth]{ch-ic/noHERA_Quad_MHOU.pdf}
  \includegraphics[width=0.49\linewidth]{ch-ic/noATLASCMSDY_Quad_MHOU.pdf}
  \includegraphics[width=0.49\linewidth]{ch-ic/nojets_Quad_MHOU.pdf}
  \includegraphics[width=0.49\linewidth]{ch-ic/noZpT_Quad_MHOU.pdf}
  \caption{\small
    Same as \cref{fig:ic/charm_dataset_dep} for the intrinsic charm (3\fns)
    \pdf (top four plots), now also including four additional dataset
    variations:  no \atlas and \cms $W, Z$ production data   (third row left),
    no jet data (third row right), no $Z$ $p_T$ measurements (bottom row left),
    no \hera \dis data (bottom row right). 
    The error band indicates the \pdf uncertainties combined in quadrature with
    the \acrshortpl{mhou}.
  }
  \label{fig:ic/charm_dataset_dep_nf3}
\end{figure}
%%%%%%%%%%%%%%%%%%%%%%%%%%%%%%%%%%%%%%%%%%%%%%%%%%%%%%%%%%%%%%%%%%%%%%


%%%%%%%%%%%%%%%%%%%%%%%%%%%%%%%%%%%%%%%%%%%%%%%%%%%%%%%%%%%%%%%%%%%
\begin{figure}
  \centering
  \includegraphics[width=0.54\linewidth]{ch-ic/Flavor_Basis_Quad_MHOU.pdf}
  \caption{\small
    Same as \cref{fig:ic/charm_basisdep} for the intrinsic (3\fns) charm.
  }
  \label{fig:ic/charm_basisdep_3FNS} 
\end{figure}
%%%%%%%%%%%%%%%%%%%%%%%%%%%%%%%%%%%%%%%%%%%%%%%%%%%%%%%%%%%%%%%%%%%%%%

%%%%%%%%%%%%%%%%%%%%%%%%%%%%%%%%%%%%%%%%%%%%%%%%%%%%%%%%%%%%%%%%%%%
\begin{figure}
  \centering
  \includegraphics[width=0.49\linewidth]{ch-ic/mass_variations_Quad_MHOU.pdf}
  \includegraphics[width=0.49\linewidth]{ch-ic/mass_variations_ratio_Quad_MHOU.pdf}
  \caption{\small      
    Same as \cref{fig:ic/charm_fitted_mcdep}, now for the intrinsic (3\fns)
    charm \pdf. Note that the intrinsic charm \pdf is scale independent. 
  }
  \label{fig:ic/mass_variations_Quad_MHOU}
\end{figure}
%%%%%%%%%%%%%%%%%%%%%%%%%%%%%%%%%%%%%%%%%%%%%%%%%%%%%%%%%%%%%%%%%%%%%%
