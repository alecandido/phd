The presence of the derivative on the left-hand-side and the convolution on the
right-hand-side turns \cref{eq:eko/dglap} into a set of coupled
integro-differential equations which are non-trivial to solve.

A possible strategy in solving \cref{eq:eko/dglap} is by tackling the problem
head-on and iteratively solve the integrals and the derivative by taking small
steps: we refer to this as \enquote{$x$-space solution}, as the solution uses
directly momentum space and this approach is adopted, e.g., by \apfel{}~\cite{Bertone:2013vaa},
\hoppet{}~\cite{Salam:2008qg}, and \qcdnum{}~\cite{Botje:2010ay}.
However, this approach becomes quite cumbersome when dealing with higher-order
corrections, as the solutions becomes more and more involved.

We follow a different strategy and apply the Mellin transformation $\Md$
\begin{equation}
    \tilde g(N) = \Md\qty[g(x)](N) = \int\limits_0^1\!\dd{x} x^{N-1} g(x)
\end{equation}
where, as well here as in the following, we denote objects in Mellin space by a
tilde.
This approach is also adopted by \pegasus{}~\cite{Vogt:2004ns} and \fk{}~\cite{Ball:2008by,Ball:2010de,DelDebbio:2007ee}.
The numerically challenging step is then shifted to the treatment of the Mellin
inverse $\Md^{-1}$, as we eventually seek for results in $x$-space (cf.\
\cref{sec:eko/theory-interpolation}).
