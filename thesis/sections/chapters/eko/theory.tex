We do not attempt to give a full review of the underlying theory
here as it is known since a long time and discussed extensive elsewhere
(see e.g.\ \cite{Peskin:1995ev,Ellis:1996mzs} and references therein).
We refer the interested reader to the specific references given in the following and to
the accompanying online documentation where instead we give a detailed
overview. All sections in the following have an equivalent section in
the online documentation. Also the respective code implementations of the
various ingredients contain relevant information and are also accessible
in the documentation via the API section.

The central equations that \eko{} is solving are the
Dokshitzer-Gribov-Lipatov-Altarelli-Parisi (\dglap) evolution
equations~\cite{Altarelli:1977zs,Gribov:1972ri,Dokshitzer:1977sg} given by
\begin{equation}
    \muF^2 \dv{\vb f}{\muF^2}{}(x,\muF^2) = \vb P (a_s(\muR^2),\muF^2) \otimes \vb f(\muF^2)
    \label{eq:eko/dglap}
\end{equation}
where $\vb f(x,\muF^2)$ is a vector of \pdf{}s over flavor space with $x$ the
momentum fraction and $\muF^2$ the factorization scale.
The main ingredients to \cref{eq:eko/dglap} are the Altarelli-Parisi splitting
functions $\vb P(a_s(\muR^2),x,\muF^2)$~\cite{Moch:2004pa,Vogt:2004mw}, which
are matrices over the flavor space.
Finally, $\otimes$ denotes the multiplicative (or Mellin) convolution.

The splitting functions $\vb P(a_s(\muR^2),x,\muF^2)$ expose a perturbative
expansion in the strong coupling $a_s(\mu_R^2)$:
\begin{equation}
    \vb P(a_s(\muR^2),x,\muF^2) = a_s(\muR^2) \vb P^{(0)}(x,\muF^2)
    + \qty[a_s(\muR^2)]^2 \vb P^{(1)}(x,\muF^2)
    + \qty[a_s(\muR^2)]^3 \vb P^{(2)}(x,\muF^2)
    + \ldots
\end{equation}
which is currently known at \nnlo{}~\cite{Moch:2004pa,Vogt:2004mw,Blumlein:2021enk} and is under
investigation for \nnnlo{}~\cite{Moch:2021qrk}.
In a first step, the renormalization scale $\muR$ and the factorization scale
$\muF$ can be assumed to be equal $\muR = \muF$ and the renormalization scale
dependence can be restored later on. The variation of the ratio $\muR/\muF$ can
be considered as an estimated to missing higher order
uncertainties (\mhou{})~\cite{AbdulKhalek:2019ihb}.

In order to solve \cref{eq:eko/dglap} a series of steps has to be taken, and we
highlight these steps in the following sections.

\subsection{Mellin space}
\label{sec:theory:mellin}
The presence of the derivative on the left-hand-side and the convolution on the
right-hand-side turns \cref{eq:eko/dglap} into a set of coupled
integro-differential equations which are non-trivial to solve.

A possible strategy in solving \cref{eq:eko/dglap} is by tackling the problem
head-on and iteratively solve the integrals and the derivative by taking small
steps: we refer to this as \enquote{$x$-space solution}, as the solution uses
directly momentum space and this approach is adopted, e.g., by \apfel{}~\cite{Bertone:2013vaa},
\hoppet{}~\cite{Salam:2008qg}, and \qcdnum{}~\cite{Botje:2010ay}.
However, this approach becomes quite cumbersome when dealing with higher-order
corrections, as the solutions becomes more and more involved.

We follow a different strategy and apply the Mellin transformation $\Md$
\begin{equation}
    \tilde g(N) = \Md\qty[g(x)](N) = \int\limits_0^1\!\dd{x} x^{N-1} g(x)
\end{equation}
where, as well here as in the following, we denote objects in Mellin space by a
tilde.
This approach is also adopted by \pegasus{}~\cite{Vogt:2004ns} and \fk{}~\cite{Ball:2008by,Ball:2010de,DelDebbio:2007ee}.
The numerically challenging step is then shifted to the treatment of the Mellin
inverse $\Md^{-1}$, as we eventually seek for results in $x$-space (cf.\
\cref{sec:eko/theory-interpolation}).


\subsection{Interpolation}
\label{sec:theory:interpolation}
To bridge between the desired $x$-space input/output and the internal
Mellin representation, we do a Lagrange-Interpolation as sketched in
\cref{sec:theory:interpolation}
(and detailed in the \href{https://eko.readthedocs.io/en/latest/}{online documentation}).
We recommend a grid of at least 50 points with
linear scaling in the large-$x$ region ($x \gtrapprox 0.1$) and with logarithmic
scaling in the small-$x$ region and an interpolation of degree four.
Also the grids determined by \amcfast~\cite{Bertone:2014zva} perform
sufficiently well for specific processes.

\begin{figure}
    \begin{center}
    \includegraphics[width=\textwidth]{ch-eko/interpolation-int-ratio}
    \end{center}
    \caption{Relative differences between 
        the outcome of \nnlo{} \qcd{} evolution
        as implemented in \eko{} with 20, 30, and 60 points to 120
        interpolation points respectively.
        \label{fig:interpolation} }
\end{figure}

For a first qualitative study we show in \cref{fig:interpolation} a
comparison between an increasing number of interpolation points
distributed according to \cite[Eq. 2.12]{Carrazza_2020}.
The separate configurations are converging to the solution with the
largest number of points. Using 60 interpolation points is almost
indistinguishable from using 120 points (the reference configuration in the plot).
In the singlet sector (gluon) the convergence is
significantly slower due to the more involved solution strategies and,
specifically, the oscillating behavior is caused due to these difficulties.
The spikes for $x\to 1$ are not relevant since the \pdfs are intrinsically
small in this region ($\vb f\to 0$) and thus small numerical differences
are enhanced.

Also note that the results of \cref{sec:pheno:bench} (i.e.\ \cref{fig:LHAbench,fig:Apfelbench_pto,fig:Pegasusbench_pto}) confirm that
the interpolation error can be kept below the benchmark accuracy.


\subsection{Strong coupling}
\label{sec:theory:coupling}
The evolution of the strong coupling $a_s(\muR^2) = \alpha_s(\muR^2)/(4\pi)$
is given by its renormalization group equation (\rge):
\begin{equation}
    \beta(a_s) = \muR^2\dv{a_s(\muR^2)}{\muR^2}{} = - \sum\limits_{n=0} \beta_n \qty[a_s(\muR^2)]^{2+n}
\end{equation}
and is currently known at 5-loop ($\beta_4$)
accuracy~\cite{Herzog:2017ohr,Luthe:2016ima,Baikov:2016tgj,Chetyrkin:2017bjc,Luthe:2017ttg}.

This is crucial for \dglap{} solution, indeed, since the strong coupling $a_s$
is a monotonic function of the renormalization scale in the perturbative
regime, we can actually consider a transformation of
\cref{eq:dglap}
\begin{equation}
    \dv{\vb{\tilde f}}{a_s}{}(N,a_s) = - \frac{\bm{\gamma}(N,a_s)}{\beta(a_s)} \vb {\tilde f}(N, a_s)
    \label{eq:dglap2}
\end{equation}
with $\bm{\gamma} = - \vb{\tilde P}$ the anomalous dimension and $\beta(a_s)$
the \qcd{} beta function, where the multiplicative convolution is reduced to an
ordinary product.


\subsection{Flavor space}
\label{sec:theory:flavor}
Next, we address the separation in flavor space: formally we can define the
flavor space $\Fd$ as the linear span over all partons (which we consider to be
the canonical one):
\begin{equation}
    \Fd = \Fd_{fl} = \vspan\qty(g, u, \bar u, d, \bar d, s, \bar s, c, \bar c, b, \bar b, t, \bar t)
\end{equation}

The splitting functions $\vb P$ become block-diagonal in the \enquote{Evolution
Basis}, a suitable decomposition of the flavor space: the singlet sector $\vb
P_S$ remains the only coupled sector over $\qty{\Sigma, g}$, while the full
valence combination $P_{ns,v}$ decouples completely (i.e.\ it is only coupling
to $V$), and the non-singlet singlet-like sector $P_{ns,+}$ is diagonal over
$\qty{T_3,T_8,T_{15},T_{24},T_{35}}$, and the non-singlet valence-like sector
$P_{ns,-}$ is diagonal over $\qty{V_3,V_8,V_{15},V_{24},V_{35}}$.
The respective distributions are given by their usual definition.

This Evolution Basis is isomorphic to our canonical choice
\begin{equation}
    \Fd \sim \Fd_{ev} = \vspan(g, \Sigma, V, T_{3}, T_{8}, T_{15}, T_{24}, T_{35}, V_{3}, V_{8}, V_{15}, V_{24}, V_{35})
\end{equation}
but, it is not a normalized basis. When dealing with intrinsic evolution, i.e.\
the evolution of \pdf{}s below their respective mass scale, the Evolution Basis
is not sufficient. In fact, for example, $T_{15} = u^{+} + d^{+} +
s^{+} - 3c^{+}$ below the charm threshold $\mu_c^2$ contains both running and static
distributions which need to be further disentangled.

We are thus considering a set of \enquote{Intrinsic Evolution Bases} $\Fd_{iev,
n_f}$, where we retain the intrinsic flavor distributions as basis vectors.
The basis definition depends on the number of light flavors $n_f$ and, e.g.\
for $n_f=4$, we find
\begin{equation}
    \Fd \sim \Fd_{iev,4} = \vspan(g, \Sigma_{(4)}, V_{(4)}, T_{3}, T_{8}, T_{15}, V_{3}, V_{8}, V_{15}, b^+, b^-, t^+, t^-)
\end{equation}
with $\Sigma_{(4)} = \sum\limits_{j=1}^4 q_j^+$ and $V_{(4)} =
\sum\limits_{j=1}^4 q_j^-$.


\subsection{Solution Strategies}
\label{sec:theory:solutions}
As already mentioned in \cref{sec:eko/theory-solutions}, due to the coupled
integro-differential structure of \cref{eq:eko/dglap},  solving the equations
requires in practice some approximations to which we refer as different
solution strategies.
\eko{} currently implements 8 different strategies, corresponding to different
approximations.
Note that they may differ only by the strategy in a specific sector, such as
the singlet or non-singlet sector. 
All the strategies provided agree at fixed order, but differ by higher order
terms.

In \cref{fig:eko/solutions} we show a comparison of a selected list of
solution strategies\footnote{
  For the full list of available solutions and a detailed descriptions see the
  \href{https://eko.readthedocs.io/en/latest}{online documentation}.
}:

\begin{itemize}
    \item \texttt{iterate-exact}:
        In the non-singlet sector we take the analytical solution
        of \cref{eq:eko/dglap2} up to the order specified.
        In the singlet sector we split the evolution path into segments
        and linearize the exponential in each segment~\cite{Bonvini:2012sh}.
        This provides effectively a straight numerical solution of \cref{eq:eko/dglap2}.
        In \cref{fig:eko/solutions} we adopt this strategy as a reference.
    \item \texttt{perturbative-exact}:
        In the non-singlet sector it coincides with \texttt{iterate-exact}.
        In the singlet sector we make an ansatz to determine the solution as a
        transformation $\vb{U}(a_s)$ of the \lo{} solution~\cite{Vogt:2004ns}. We then
        iteratively determine the perturbative coefficients of $\textbf{U}$.
    \item \texttt{iterate-expanded}:
        In the singlet sector we follow the strategy of \texttt{iterate-exact}.
        In the non-singlet sector we expand \cref{eq:eko/dglap2} first to the order
        specified, before solving the equations.
    \item \texttt{truncated}: 
        In both sectors, singlet and non-singlet, we make an ansatz to determine the solution as a
        transformation $\vb{U}(a_s)$ of the \lo{} solution and
        then expand the transformation $\vb U$ up to the order specified.
        Note that for programs using $x$-space this strategy is difficult
        to pursue as the \lo{} solution is kept exact and only the transformation
        $\vb U$ is expanded.
\end{itemize}

The strategies differ most in the small-$x$ region where the \pdf{} evolution is
enhanced and the treatment of sub-leading corrections become relevant.
This feature is seen most prominently in the singlet sector between
\texttt{iterate-exact} (the reference strategy) and \texttt{truncated}.
In the non-singlet sector the distributions also vanish for small-$x$
and so the difference gets artificially enhanced.
This is eventually the source of the spread for the valence distribution $V(x)$
making it more sensitive to the initial \pdf{}.

\begin{figure}
    \centering
    \includegraphics[width=\textwidth]{ch-eko/solutions-main}
    \caption{Compare selected solutions strategies, with respect to the
        \texttt{iterated-exact} (called \texttt{exa} in label) one. In
        particular: \texttt{perturbative-exact} (\texttt{pexa}) (matching
        the reference in the non-singlet sector),
        \texttt{iterated-expanded} (\texttt{exp}), and \texttt{truncated}
        (\texttt{trn}).
        The distributions are evolved in $\muF^2=\SI[parse-numbers=false]{1.65^2\to 10^4}{\GeV^2}$.}
    \label{fig:eko/solutions}
\end{figure}

\paragraph{PDF plots} The \pdf{} plot shown in \cref{fig:eko/solutions} contains
multiple elements, and its layout is in common with
\cref{fig:eko/interpolation,fig:eko/pdfmatching}.

All the different entries corresponds to different theory settings, and they
are normalized with respect to a reference theory setup
(e.g.\ in \cref{fig:eko/solutions} the \texttt{iterative\allowbreak-exact}
strategy) and the lines correspond to the relative difference.

Furthermore, an envelope and dashed lines are displayed.
To obtain them, the full \pdf{} set is evolved, replica by replica, for each
configuration (corresponding to a single evolution operator, that is applied to
each replica in turn).
Then ratios are taken between corresponding evolved replicas, to highlight
the \pdf{} independence of \eko{} rather then any specific set-related features.
The upper and lower borders of the envelope correspond respectively
to the $0.16$ and $0.84$ quantiles of the replicas set,
while the dashed lines correspond to one standard deviation.


\subsection{Matching at Thresholds}
\label{sec:theory:matching}
\eko{} can perform calculation in a fixed flavor number scheme (\ffns{}) where
the number of active or light flavors $n_f$ is constant. This means that both
the beta function $\beta^{(n_f)}(a_s)$ and the anomalous dimension
$\bm{\gamma}^{(n_f)}(a_s)$ in \cref{eq:eko/dglap2} are constant with respect to
$n_f$.
However, this approximation is likely to fail either in the high energy region
$\muF^2 \to \infty$ for a small number of active flavors, or to fail in the low
energy region $\muF^2 \to \Lambda_{\text{QCD}}^2$ for a large number of active
flavors.

This can be overcome by using a variable flavor number scheme (\vfns{}) that
changes the number of active flavors when the scale $\muF^2$ crosses a
threshold $\mu_h^2$.
This then requires a matching procedure when changing the number of active
flavors, and for the \pdfs we find
\begin{align}
  \tilde{\mathbf{f}}^{(n_f+1)}(\mu_{F,1}^2)=&~ \tilde{\mathbf{E}}^{(n_f+1)}(\mu_{F,1}^2\leftarrow \mu_{h}^2) {\mathbf{R}^{(n_f)}} 
  \tilde{\mathbf{A}}^{(n_f)}(\mu_{h}^2) \tilde{\mathbf{E}}^{(n_f)}(\mu_{h}^2\leftarrow \mu_{F,0}^2)\nonumber\\
  &\qquad \times\tilde{\mathbf{f}}^{(n_f)}(\mu_{F,0}^2)
  \label{eq:eko/matching}
\end{align}
where the superscript refers to the number of active flavors and we split the
matching factor into two parts: the perturbative \acrfull{ome}
$\tilde{\mathbf{A}}^{(n_f)}(\mu_{h}^2)$, currently implemented at
\nnlo{}~\cite{Buza_1998}, and an algebraic rotation ${\mathbf{R}^{(n_f)}}$
acting only in the flavor space $\Fd$.

For backward evolution this matching has to be applied in the reversed order.
The inversion of the basis rotation matrices $\mathbf{R}^{(n_f)}$ is simple,
whereas this is not true for the \ome $\mathbf{\tilde A}^{(n_f)}$ especially
in case of \nnlo{} or higher order evolution.
In \eko{} we have implemented two different strategies to perform the inverse
matching: the first one is a numerical inversion, where the OMEs are inverted
exactly in Mellin space, while in the second method, called \texttt{expanded},
the matching matrices are inverted through a perturbative expansion in $a_s$,
given by:
\begin{align}
  \qty(\mathbf{\tilde A}^{(n_f)})_{exp}^{-1}(\mu_{h}^2) = \mathbf{I}
  &- a_s(\mu_{h}^2) \mathbf{\tilde A}^{(n_f),(1)} \nonumber\\
  &+ a_s^2(\mu_{h}^2) \qty[ \mathbf{\tilde A}^{(n_f),(2)} - \qty(\mathbf{\tilde A}^{(n_f),(1)})^2 ] \nonumber\\
  &+ O(a_s^3)
  \label{eq:eko/invmatchingexp}
\end{align}
with $\mathbf{I}$ the identity matrix in flavor space.


\subsection{Running Quark Masses}
\label{sec:theory:msbarmass}
In the context of \pdf{} evolution, the most used treatment of heavy quarks masses are the pole masses,
where the physical values are specified as input and do not depend on any scale.
However for specific applications, such as the determination of \mhou{} due to heavy quarks contribution 
inside the proton~\cite{Ball:2016neh}, \msbar{} masses can also be used.
In particular, in~\cite{Alekhin:2010sv} it is found that higher order corrections on heavy quark production
in \dis{} are more stable upon scale variation when using the \msbar{} scheme.
\eko{} allows for this option as it is discussed briefly in the following paragraphs.

Whenever the initial condition for the mass is not given at a scale coinciding with
the mass itself (i.e. $\mu_{h,0} \neq m_{h,0}$, being $m_{h,0}$ the given initial condition
at the scale $\mu_{h,0}$),
\eko{} computes the scale at which the running mass $m_{h}(\mu_h^2)$ intersects
the identity function.
Thus for each heavy quark $h$ we solve:
%
\begin{equation}
    m_{\overline{MS},h}(m_h^2) = m_h
\end{equation}
The value $m_h(m_h)$ is then used as a reference to define the evolution thresholds.

The evolution of the \msbar{} mass is given by:
%
\begin{equation}
    m_{\overline{MS},h}(\mu_h^2) = m_{h,0} \exp\qty[ - \int\limits_{a_s(\mu_{h,0}^2)}^{a_s(\mu_h^2)} \frac{\gamma_m(a_s')}{\beta(a_s')} \dd{a_s'} ]
    \label{eq:eko/msbarsolution}
\end{equation}
%
with $\gamma_m(a_s)$ the \qcd{} anomalous mass dimension available up to 
\nnnlo{}~\cite{Vermaseren:1997fq,Schroder:2005hy,Chetyrkin:2005ia}.

Note that to solve \cref{eq:eko/msbarsolution} $a_s(\mu_R^2)$ must be evaluated in 
a \ffns{} until the threshold scales are known. Thus it is important
to start computing the \msbar{} masses of the quarks which are closer to the
the scale $\mu_{R,0}$ at which the initial reference value $a_s(\mu_{R,0}^2)$ is given. 

Furthermore, to find consistent solutions the boundary condition of the
\msbar{} masses must satisfy $m_h(\mu_h) \ge \mu_h$ for heavy quarks involving
a number of active flavors greater than the number of quark flavors $n_{f,0}$ at $\mu_{R,0}$, implying that we find
the intercept between the \rge{} and the identity in the forward direction ($m_{\overline{MS},h} \ge \mu_h$).
The opposite holds for scales related to fewer active flavors.

