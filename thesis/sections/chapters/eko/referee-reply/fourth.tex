\documentclass[a4paper,11pt]{article}
% \pdfoutput=1 % if your are submitting a pdflatex (i.e. if you have
             % images in pdf, png or jpg format)

\usepackage{../styles/jheppub} % for details on the use of the package, please
                     % see the JHEP-author-manual

\usepackage[T1]{fontenc} % if needed
\usepackage{lmodern}

\usepackage{../styles/main}
\usepackage{../styles/defs}


\usepackage{microtype,xparse,tcolorbox}
\newenvironment{reviewer-comment }{}{}
\tcbuselibrary{skins}
\tcolorboxenvironment{reviewer-comment }{empty,
  left = 1em, top = 1ex, bottom = 1ex,
  borderline west = {2pt} {0pt} {black!20},
}
\newcounter{comment}[section]
\ExplSyntaxOn
\NewDocumentEnvironment {response} { +m O{black!20} } {\refstepcounter{comment}
  \IfValueT {#1} {
    \begin{reviewer-comment~}
      \noindent
      \textbf{Comment\hspace{1mm}\thecomment :\hspace{2mm}} \ttfamily #1 
    \end{reviewer-comment~}
  }
  \par\noindent\ignorespaces
} { \bigskip\par\par }
\ExplSyntaxOff


\begin{document} 
%\maketitle
%\flushbottom


\begin{response}{
  The authors have added a more complete usage example in Appendix B.2, as
requested. I am going to make a few more suggestions. While I value, and do not
shy away  from, the task of a referee to provide advice to the authors, these
multiple iterations concerning fairly technical or trivial details seem to me to
be, to quote a fellow physicist, a suboptimal use of my time. The authors should
please try to get it right on the next iteration, putting themselves in the
shoes of a prospective user of their code.
}
We thank the referee for reviewing our manuscript and providing suggestions / comments for improvement.

In the following, we arrange the referee's comments into a series of items and 
will address them one by one.
\end{response}

\begin{response}{
  The example in Appendix B.2 runs correctly with my Python 3.9 installation.
  However, it depends on a successful  installation of LHAPDF and its Python
  interface. This is not always a trivial task. For instance, I could install it
  on my oldish Mac using versions 6.3.0 and 6.5.1, but not 6.4.0. This sensitivity
  to LHAPDF versions is of course not the authors' fault or responsibility, but it
  suggests that including LHAPDF in what should by a minimal ``quick-start''
  example is probably not a good idea.
}
We agree with the referee that indeed the LHAPDF management is not our responsibility, but
instead should be discussed with the LHAPDF authors. However, we would like to point out that
we are not dependent on the actual version (and e.g.\ one of us is running successfully LHAPDF with version 6.4.0).
Still, any prospective user is most likely using the LHAPDF interface. In this context we like to point out again
that EKO is designed as a library and not as a program: there is no explicit dependence on LHAPDF inside EKO since the goal of the
library is to be PDF independent.

Concerning the example, we adjusted the example given in the paper to redo the official benchmark described in
Ref.~[45]. We would like to recall that the benchmark settings are in fact the only possibility to do
PDF evolution without LHAPDF, in a common context (custom functions are possible, but not very meaningful, nor user-friendly).
\end{response}

\begin{response}{
  Moreover, after managing to run the example using LHAPDF I found that,
  after evolution with EKO, I could not reproduce the pdf values directly
  obtained from the LHAPDF package, even after adjusting to the best of
  my understanding all EKO parameters.
}
While the actual example is discussed in the reply to Comment 5, we briefly review here the benchmarking
process that was established during the development of EKO.

First, we mention that the EKO development has been conducted with an explicit emphasis on benchmarking, since
several other evolution program were already available, and adopted by the community. Therefore, a significant
amount of time has been invested in this part.
In fact, we developed a full benchmarking system (supported by the \texttt{banana-hep} package) that we use
routinely to check our code against other codes.
This benchmarking system, however, is not discussed in the paper, since the paper is just presenting the EKO library, 
while the benchmarking system is mostly intended for internal use (as its use requires an advanced
installation from the repository).

Amongst the available benchmarks, we explicitly provide a system to benchmark to arbitrary PDF sets available
through LHAPDF (currently e.g.\ to CT18 and NNPDF4.0). However, as is well known, the usage of the LHAPDF
interface has limited precision in the border region of the respective PDF grid. Also the problems discussed in
Comment 2 can be a limiting factor. Instead, a stronger check can be established by comparing directly to other
evolution codes.

As mentioned in the reply to Comment 2, the benchmarks in Refs.~[45,46] are usually considered a sufficient check
for a new evolution code (and, in fact, all other evolution programs so far did pass that check - consider that
HOPPET and PEGASUS where used to established the benchmark): we present this check in Fig.~1.
As can be seen explicitly in this plot the official benchmark only consists of a few point. Indeed, a stronger
check can be imposed by directly comparing to other evolution codes as is done here for PEGASUS and APFEL which instead
are compared on a significant larger number of points. This limitation of precision due to limited number of points
also applies to any LHAPDF set which on top applies an additional interpolation.

We like to stress that using our benchmarking system we can do (and repeat) a benchmark with arbitrary settings and arbitrary boundary
conditions to any program, provided only, of course, both programs support the settings. To the best of our knowledge this is
the first implementation of such a benchmarking system (which in effect can be used with a single command) and that is very useful
in achieving the goal of reproducibility.

We would also like to point out that our strongest check is the successful recomputation of all datasets used in the NNPDF4.0
analysis~[5], since this tests a huge kinematic range - see Fig.~2.1 of Ref.~[5].

In addition, on a more technical level, we also like to point out that EKO implements a significant amount of unit tests
(at the time of writing 216) that check atomic properties of code, mathematical properties or invariants. These tests do cover
100\% of the EKO code basis.
\end{response}

\begin{response}{
  I therefore suggest that the authors:
  
  1. provide a fully self-contained quick-start example, as well as its expected
  output. I could for instance produce one using the toy PDF provided via "from
  banana import toy". In the same example, they may wish to be a bit more generous
  in showing how to access both the initial and the evolved PDF, also considering
  that the online documentation is not very helpful.
}
As said in the reply to Comment 2, we adjusted the example in the paper to the official benchmark
using the toy PDF as defined by Ref.~[45] and as implemented in \texttt{banana}.
The extension to the full benchmark described in Ref.~[45] is simple.

As said in the example the \texttt{pdf} variable is a LHAPDF-like object and in the previous revision
it was actually a true LHAPDF object. Instead, \texttt{toy.mkPDF} only implements the fundamental methods
such as \texttt{xfxQ2} to query for the initial PDF (for a full list, please refer to the respective online documentation).
We like to recall explicitly that the toy PDF is only defined at the initial scale (see definition in Ref. [45]),
so it doesn't depend on any $Q^2$ scale.

Concerning the access to the evolved PDF we repeat the relevant part of the snippet
\begin{lstlisting}[language=Python]
# then e.g. print the evolved gluon (pid = 21) at Q2 = 10000 GeV^2 for the first point x=1e-7
print(evolved_pdfs[10000.]["pdfs"][21][0])
\end{lstlisting}
and we point out explicitly that the first index to \texttt{evolved\_pdfs} refers to the final scale (here \texttt{10000.}),
the second index is to be kept fixed,
the third index refers to the Monte Carlo parton identifier (here \texttt{21} which means gluon), and the fourth index refers to the index
in the interpolation grid (here \texttt{0} which refers to \texttt{1e-7}).
Note that the final PDF can only be accessed on the interpolation grid (as this is the purpose of the library) and to overcome
this limitation we provide also a re-export to LHAPDF as is discussed in the online documentation\footnote{\url{https://eko.readthedocs.io/en/latest/overview/tutorials/pdf.html\#Method-2:-Using-ekobox}}.
\end{response}

\begin{response}{  
  2. provide an example that uses LHAPDF, and explicitly shows that one
  can reproduce, using EKO, the values of the pdfs that one can obtain
  from the LHAPDF package. This second example may also go in the online
  documentation if the authors do not wish to make the paper longer.
}
As suggested, we added an explicit example on how to compare to an existing PDF set in the online documentation\footnote{\url{https://eko.readthedocs.io/en/latest/overview/tutorials/pdf.html\#A-more-Realistic-Example:-Benchmark-to-CT14llo}}.
\end{response}

\begin{response}{
  3. Speaking of the online documentation, I'm afraid it's still in quite
  a sorry state. If not incomplete, it's at least very difficult to
  navigate, one does not have a real sense of what the main methods are,
  if the list of the API is complete, what are the parameters to be set
  (and/or their best values, if they can be determined). To give just one
  example of the incompleteness of the online documentation it suffices
  to look at https://eko.readthedocs.io/en/latest/code/IO.html. If one
  wanted to know what PTO value to use for NNLO evolution, one could only
  guess it by extrapolating it to PTO=2. The list of the parameters is,
  of course, also largely incomplete, as the full content of th\_card in
  the example shows. My opinion is that this online documentation should
  also be improved before publication, and not asymptotically after it.
  If improving it across the board is too big a task, at least make it
  more useful by including a few full and concrete examples, that make
  use of the most important methods.
}
Once again, we would like to point out that the content in the paper describes the version in the paper,
whereas the online documentation refers to the current version (which is not the version in the paper).

We also ensure the referee that the API is in fact complete and describes all public visible methods
(note that the concept of visibility in Python is different from e.g.\ C++).
Moreover, to the best of our knowledge, we are the only evolution program that provides such a complete list
and, in addition, since we provide the list in an online documentation we can and do provide many cross links.

The main method to use is the one described in the example (again we stress that EKO is a library).
The use of any additional method is mainly intended for other programmers that develop or use evolution codes, such as
anomalous dimensions which can simply be found under this exact name in the API. We stress this is not for the common
user.

The list of parameters to set for the paper version is exactly the list given in the paper. As we explain in the paragraph
before the example, we do not provide any default value, but instead leave it to the users responsibility to choose
the correct and consistent values suitable for his application. We further clarified in the example that indeed by
\enquote{perturbative order} we mean perturbative order of evolution and that the mapping 0=LO, 1=NLO, 2=NNLO holds.

Instead, for the online documentation we updated the page mentioned by the referee with a full list and description of
all parameters. We stress that the list is a complete list of the relevant parameters and any additional parameters
that might be present in the theory card are simply ignored (e.g.\ masses of gauge bosons or CKM matrix elements).
The presence of these additional parameters is simply due to the fact that we are bound to provide an interface to the
NNPDF ecosystem (see e.g.\ Ref.~[5] and references therein).

\end{response}

% \bibliographystyle{../styles/JHEP}
% \bibliography{../bibliography/eko.bib,../bibliography/refs.bib}

% \listoffixmes

\end{document}
