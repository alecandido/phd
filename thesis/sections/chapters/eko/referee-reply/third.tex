\documentclass[a4paper,11pt]{article}
\pdfoutput=1 % if your are submitting a pdflatex (i.e. if you have
             % images in pdf, png or jpg format)

\usepackage{../styles/jheppub} % for details on the use of the package, please
                     % see the JHEP-author-manual

\usepackage[T1]{fontenc} % if needed
\usepackage{lmodern}

\usepackage{../styles/main}
\usepackage{../styles/defs}


\usepackage{microtype,xparse,tcolorbox}
\newenvironment{reviewer-comment }{}{}
\tcbuselibrary{skins}
\tcolorboxenvironment{reviewer-comment }{empty,
  left = 1em, top = 1ex, bottom = 1ex,
  borderline west = {2pt} {0pt} {black!20},
}
\newcounter{comment}[section]
\ExplSyntaxOn
\NewDocumentEnvironment {response} { +m O{black!20} } {\refstepcounter{comment}
  \IfValueT {#1} {
    \begin{reviewer-comment~}
      \noindent
      \textbf{Comment\hspace{1mm}\thecomment :\hspace{2mm}} \ttfamily #1 
    \end{reviewer-comment~}
  }
  \par\noindent\ignorespaces
} { \bigskip\par\par }
\ExplSyntaxOff


\begin{document} 
%\maketitle
%\flushbottom


\begin{response}{
  The authors have performed minor corrections to the text of the paper and, more
importantly, added an appendix B where they show explicitly how to get started
with their code. This addition is welcome and answers at least in part my
concerns.

Here follow some comments about the changes.

}

We thank the referee for reviewing our manuscript and providing suggestions / comments for improvement.

In the following, we arrange the referee's comments into a series of items and 
will address them one by one.
\end{response}

\begin{response}{
  Concerning footnote "a" added on page 18, I may speculate that the reason why
  APFEL and HOPPET do not provide an interface to write out and read in the
  evolution operator is that it's probably faster to recalculate it on the fly.
  I'm not suggesting that the authors make changes, but I would advise not to
  stress too much this point as a supposed advantage of EKO with respect to those
  packages.
}
As we explain in the second paragraph of that page and Appendix A they are indeed faster for the evolution of a single \pdf{}.
Instead, with \eko{} we address the case of \pdf{} fitting where a repeated evolution is needed and where we gain an advantage.
Moreover, we do need a dedicated interface to produce \fk{} tables.
\end{response}

\begin{response}{
  The addition of appendix B is most welcome. I was able to install and,
  eventually, run the code. I think this \enquote{quick start} guide is however still too
  minimal. A user is left with an \enquote{evolution\_operator} but no way to know what to
  do with it. He or she is referred to an online documentation, which however is
  not only partially incompatible (e.g.\ different th\_card, different op\_card), but
  also not much more complete. \enquote{Explore the output} also leaves the user without
  an explicit way to access to the content of \enquote{evolution\_operator} (I should
  also mention that I could extract nothing but zeros from it. But since I am no
  python expert, I won't exclude an error on my part).
  
  I think that Appendix B, and the online documentation, should include at least a
  full example, e.g. giving a PDF and performing its evolution with a calculated
  operator, outputting in the meantime the alpha\_strong values at the relevant
  scales. Such example should come with input cards that are sufficiently
  documented that one knows what each parameter refers to, and an output that one
  can compare his or her own to.
}
We adjusted Appendix B to explicitly evolve a \pdf{} and output the value of $\alpha_s$.
We added a brief description to each entry of the theory setting and the operator settings.
We also added an explicit sentence to clarify that Appendix B only refers to the version released
together with this paper (\texttt{v0.8.5}).
Instead, for the instructions for the current version we refer to the online documentation which we adjusted in the same way.

We can reassure the referee that the paper does describe the \eko{} library and the evolution
operator contains non-zero elements in such a way to match, e.g., the Les Houches benchmark (see Fig. 1).
\end{response}

\begin{response}{
  I can recommend publication of the paper after such a minimal but complete
  example has been provided.
  
  The authors are also certainly aware that the online documentation still needs
  quite some work. There is, for instance, very little curation showing what the
  most  important methods are, how to best perform at least the basic tasks, etc.\
  I invite them to put some effort in improving this important resource.
}
We agree with the referee that a good documentation is a very valuable document and this statement was one of the main motivations
to develop \eko{}. However, the library was, is, and for some time will be under development and so we are hesitant to document
a syntax which we know is temporary. In this respect we stress again that the paper describes the version released with it
and future versions are expected to expose a different behavior.
We keep a dedicated effort to add new documentation upon the development of new features or code refactor.
\end{response}

% \bibliographystyle{../styles/JHEP}
% \bibliography{../bibliography/eko.bib,../bibliography/refs.bib}

% \listoffixmes

\end{document}
