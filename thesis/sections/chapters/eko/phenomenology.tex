Although \eko{} is totally \pdf{} independent, for the sake of plotting
we choose NNPDF4.0~\cite{NNPDF:2021njg} as a default choice for
our plots, but for \cref{sec:eko/pheno-bench} where we choose the toy \pdf{} of the
Les Houches Benchmarks~\cite{Giele:2002hx,Dittmar:2005ed}.
We show the gluon distribution $g(x)$ as a
representative member of the singlet sector and the valence distribution $V(x)$
as a representative member of the non-singlet sector.
Note that \pdfs in the same sector have mostly the same behavior, apart from
some specific regions (e.g.\ the $T_{15}$ distribution right after charm
matching).

\subsection{Benchmarks}
\label{sec:eko/pheno-bench}
In this section we present the outcome of the benchmarks between \eko{} and similar 
available tools assuming different theoretical settings.

\subsubsection{Les Houches Benchmarks}
\eko{} has been compared with the benchmark tables
given in \cite{Giele:2002hx,Dittmar:2005ed}.
We find a good match except for a list of typos which we list here:
\begin{itemize}
    \item in table head in \cite{Giele:2002hx} should be $2xL_+ = 2x(\bar u + \bar d)$
    \item in the head of table 1: the value for $\alpha_s$ in \ffns{} is wrong (as pointed out and corrected in \cite{Dittmar:2005ed})
    \item in table 3, part 3 of \cite{Giele:2002hx}: $xL_-(x=10^{-4}, \muF^2 = \SI{1e4}{\GeV^2})=1.0121\cdot 10^{-4}$ (wrong exponent) and
          $xL_-(x=0.1, \mu_F^2 = \SI{1e4}{\GeV^2})=9.8435\cdot 10^{-3}$ (wrong exponent)
    \item in table 15, part 1 of \cite{Dittmar:2005ed}: $xd_v(x=10^{-4}, \mu_F^2 = \SI{1e4}{\GeV^2}) = 1.0699\cdot 10^{-4}$ (wrong exponent) and
          $xg(x=10^{-4}, \mu_F^2 = \SI{1e4}{\GeV^2}) = 9.9694\cdot 10^{2}$ (wrong exponent)
\end{itemize}
Some of these typos have been already reported in \cite{Diehl:2021gvs}.

\begin{figure}
    \centering
    \includegraphics[width=\textwidth,height=.24\textheight]{ch-eko/lha_bench_g_S}
    \includegraphics[width=\textwidth,height=.24\textheight]{ch-eko/lha_bench_T3_T8}
    \includegraphics[width=\textwidth,height=.24\textheight]{ch-eko/lha_bench_T15_T24}
    \includegraphics[width=\textwidth,height=.24\textheight]{ch-eko/lha_bench_V_V3}
    \caption{Relative differences between 
        the outcome of \nnlo{} \qcd{} evolution
        as implemented in \eko{} and the
        corresponding results from \cite{Dittmar:2005ed}, \apfel{}~\cite{Bertone:2013vaa} and \pegasus{}~\cite{Vogt:2004ns}.
        We adopt the settings of the Les Houches \pdf{} evolution benchmarks~\cite{Giele:2002hx,Dittmar:2005ed}.}
    \label{fig:eko/LHAbench}
\end{figure}

In \cref{fig:eko/LHAbench} we present the results of the \vfns{} benchmark at
\nnlo{}, where a toy \pdf{} is evolved from $\mu_{F,0}^2=\SI{2}{\GeV^2}$ up to
$\mu_{F}^2=\SI{1e4}{\GeV^2}$ with equal values of the factorization and process
scales $\muF=\mu$.
For completeness, we display the singlet $S(x)$ and gluon $g(x)$ distribution
(top), the singlet-like $T_{3,8,15,24}(x)$ (middle) and the valence $V(x)$,
valence-like $V_{3}(x)$ (bottom) along with the results from \apfel{} and
\pegasus{}. We find an overall agreement at the level of $O(10^{-3})$.


\subsubsection{APFEL}
\apfel{} \cite{Bertone:2013vaa} is one of the most extensive tool aimed to
\pdf{}  evolution and \dis{} observables calculation.
It is provided as a Fortran library, and it has been used by the NNPDF
collaboration up to NNPDF4.0~\cite{NNPDF:2021njg}.

\apfel{} solves \dglap{} numerically in $x$-space, sampling the evolution
kernels on a grid of points up to \nnlo{} in \qcd{}, with \qed{} evolution also
available at \lo{}.
By construction this method is \pdf{} dependent and the code is automatically
interfaced with \lhapdf{}~\cite{Buckley:2014ana}. For specific application,
the code offers also the possibility to retrieve the evolution operators
with a dedicated function.

The program supplies three different solution strategies, with various theory
setups, including scale variations and \msbar{} masses.

The stability of our benchmark at different perturbative orders is presented in \cref{fig:eko/Apfelbench_pto},
using the settings of the Les Houches \pdf{} evolution benchmarks~\cite{Giele:2002hx,Dittmar:2005ed}.
The accuracy of our comparison is not affected by the increasing complexity
of the calculation.

\begin{figure}
    \includegraphics[width=\linewidth]{ch-eko/Apfel_bench_pto.png}
    \caption{Relative differences between the outcome of evolution as
        implemented in \eko{} and the corresponding results from \apfel{} at
        different perturbative orders.  We adopt the same settings of
        \cref{fig:eko/LHAbench}.}
    \label{fig:eko/Apfelbench_pto}
\end{figure}


\subsubsection{PEGASUS}
\pegasus{}~\cite{Vogt:2004ns} is a Fortran program aimed for \pdf{} evolution.
The program solves \dglap{} numerically in $N$-space up to \nnlo{}.
\pegasus{} can only deal with \pdfs given as a fixed functional form and is
not interfaced with \lhapdf{}.

As shown in \cref{fig:LHAbench}, the agreement of \eko{} with this tool is better than with \apfel{},
especially for valence-like quantities, for which an exact solution is possible, where we reach
$\mathcal{O}(10^{-6})$ relative accuracy.
This is expected and can be traced back to the same \dglap{} solution strategy in Mellin space.

Similarly to the \apfel{} benchmark, we assert that the precision of our benchmark with \pegasus{} is not affected
by the different \qcd{} perturbative orders, as visible in \cref{fig:Pegasusbench_pto}.
As both, \apfel{} and \pegasus{}, have been benchmarked against
\hoppet{}~\cite{Salam:2008qg} and \qcdnum{}~\cite{Botje:2010ay} we conclude
to agree also with these programs.

\begin{figure}
    \includegraphics[width=\linewidth]{ch-eko/Pegasus_bench_pto.pdf}
    \caption{Same of \cref{fig:Apfelbench_pto}, now comparing to \pegasus{}~\cite{Vogt:2004ns}.
        \label{fig:Pegasusbench_pto} }
\end{figure}


%\subsubsection{Other tools}
%Finally we have benchmarked \eko{} comparing to other PDFs by means of \lhapdf{}.
In \cref{fig:eko/Hera_ct18_bench} we report a comparison of the evolution of
HERAPDF20 and CT18 at \nnlo{} from $\mu_{F}^2=\mu_{0}^2 \rightarrow 10^4~GeV$, where
$Q_{0}$ is the relative fitting scale.

During the fitting procedure two PDFs sets are evolved from the fitting scales
with different tools: respectively \hoppet{} and \qcdnum{}, while both collaborations used
\xfitter{} as a minimizer.

The benchmark with respect to HERAPDF20 indicates an agreement with \eko{}
both for singlet and valence-like quantities with a relative accuracy of $\mathcal{O}(10^{-3})$.
On the other hand the comparison with CT18 is more subtle and some discrepancies are visible
in the low-x region for valence-like quantities. We remark that even in this where the relative accuracy
deteriorate quickly, the absolute difference remains under control so the benchmark can be considered
acceptable.

\gmerror{Do we want to keep this benchmark?? Expand explanation ?}

\begin{figure}
      \includegraphics[width=\linewidth]{ch-eko/hera_ct18_bench.pdf}
      \caption{Relative (top)  and absolute (bottom) differences between
          the outcome of \nnlo{} \qcd{} evolution
          as implemented in \eko{} and the
          corresponding results from \lhapdf{} for HERAPDF20 and CT18.
          The evolution range is taken from the fitting scale to $\mu_{F}^2=10^4~GeV$.
          \label{fig:eko/Hera_ct18_bench} }
  \end{figure}



\subsection{Solution Strategies}
\label{sec:eko/pheno-sols}
As already mentioned in \cref{sec:eko/theory-solutions}, due to the coupled
integro-differential structure of \cref{eq:eko/dglap},  solving the equations
requires in practice some approximations to which we refer as different
solution strategies.
\eko{} currently implements 8 different strategies, corresponding to different
approximations.
Note that they may differ only by the strategy in a specific sector, such as
the singlet or non-singlet sector. 
All the strategies provided agree at fixed order, but differ by higher order
terms.

In \cref{fig:eko/solutions} we show a comparison of a selected list of
solution strategies\footnote{
  For the full list of available solutions and a detailed descriptions see the
  \href{https://eko.readthedocs.io/en/latest}{online documentation}.
}:

\begin{itemize}
    \item \texttt{iterate-exact}:
        In the non-singlet sector we take the analytical solution
        of \cref{eq:eko/dglap2} up to the order specified.
        In the singlet sector we split the evolution path into segments
        and linearize the exponential in each segment~\cite{Bonvini:2012sh}.
        This provides effectively a straight numerical solution of \cref{eq:eko/dglap2}.
        In \cref{fig:eko/solutions} we adopt this strategy as a reference.
    \item \texttt{perturbative-exact}:
        In the non-singlet sector it coincides with \texttt{iterate-exact}.
        In the singlet sector we make an ansatz to determine the solution as a
        transformation $\vb{U}(a_s)$ of the \lo{} solution~\cite{Vogt:2004ns}. We then
        iteratively determine the perturbative coefficients of $\textbf{U}$.
    \item \texttt{iterate-expanded}:
        In the singlet sector we follow the strategy of \texttt{iterate-exact}.
        In the non-singlet sector we expand \cref{eq:eko/dglap2} first to the order
        specified, before solving the equations.
    \item \texttt{truncated}: 
        In both sectors, singlet and non-singlet, we make an ansatz to determine the solution as a
        transformation $\vb{U}(a_s)$ of the \lo{} solution and
        then expand the transformation $\vb U$ up to the order specified.
        Note that for programs using $x$-space this strategy is difficult
        to pursue as the \lo{} solution is kept exact and only the transformation
        $\vb U$ is expanded.
\end{itemize}

The strategies differ most in the small-$x$ region where the \pdf{} evolution is
enhanced and the treatment of sub-leading corrections become relevant.
This feature is seen most prominently in the singlet sector between
\texttt{iterate-exact} (the reference strategy) and \texttt{truncated}.
In the non-singlet sector the distributions also vanish for small-$x$
and so the difference gets artificially enhanced.
This is eventually the source of the spread for the valence distribution $V(x)$
making it more sensitive to the initial \pdf{}.

\begin{figure}
    \centering
    \includegraphics[width=\textwidth]{ch-eko/solutions-main}
    \caption{Compare selected solutions strategies, with respect to the
        \texttt{iterated-exact} (called \texttt{exa} in label) one. In
        particular: \texttt{perturbative-exact} (\texttt{pexa}) (matching
        the reference in the non-singlet sector),
        \texttt{iterated-expanded} (\texttt{exp}), and \texttt{truncated}
        (\texttt{trn}).
        The distributions are evolved in $\muF^2=\SI[parse-numbers=false]{1.65^2\to 10^4}{\GeV^2}$.}
    \label{fig:eko/solutions}
\end{figure}

\paragraph{PDF plots} The \pdf{} plot shown in \cref{fig:eko/solutions} contains
multiple elements, and its layout is in common with
\cref{fig:eko/interpolation,fig:eko/pdfmatching}.

All the different entries corresponds to different theory settings, and they
are normalized with respect to a reference theory setup
(e.g.\ in \cref{fig:eko/solutions} the \texttt{iterative\allowbreak-exact}
strategy) and the lines correspond to the relative difference.

Furthermore, an envelope and dashed lines are displayed.
To obtain them, the full \pdf{} set is evolved, replica by replica, for each
configuration (corresponding to a single evolution operator, that is applied to
each replica in turn).
Then ratios are taken between corresponding evolved replicas, to highlight
the \pdf{} independence of \eko{} rather then any specific set-related features.
The upper and lower borders of the envelope correspond respectively
to the $0.16$ and $0.84$ quantiles of the replicas set,
while the dashed lines correspond to one standard deviation.


\subsection{Interpolation}
\label{sec:eko/pheno-interp}
To bridge between the desired $x$-space input/output and the internal
Mellin representation, we do a Lagrange-Interpolation as sketched in
\cref{sec:theory:interpolation}
(and detailed in the \href{https://eko.readthedocs.io/en/latest/}{online documentation}).
We recommend a grid of at least 50 points with
linear scaling in the large-$x$ region ($x \gtrapprox 0.1$) and with logarithmic
scaling in the small-$x$ region and an interpolation of degree four.
Also the grids determined by \amcfast~\cite{Bertone:2014zva} perform
sufficiently well for specific processes.

\begin{figure}
    \begin{center}
    \includegraphics[width=\textwidth]{ch-eko/interpolation-int-ratio}
    \end{center}
    \caption{Relative differences between 
        the outcome of \nnlo{} \qcd{} evolution
        as implemented in \eko{} with 20, 30, and 60 points to 120
        interpolation points respectively.
        \label{fig:interpolation} }
\end{figure}

For a first qualitative study we show in \cref{fig:interpolation} a
comparison between an increasing number of interpolation points
distributed according to \cite[Eq. 2.12]{Carrazza_2020}.
The separate configurations are converging to the solution with the
largest number of points. Using 60 interpolation points is almost
indistinguishable from using 120 points (the reference configuration in the plot).
In the singlet sector (gluon) the convergence is
significantly slower due to the more involved solution strategies and,
specifically, the oscillating behavior is caused due to these difficulties.
The spikes for $x\to 1$ are not relevant since the \pdfs are intrinsically
small in this region ($\vb f\to 0$) and thus small numerical differences
are enhanced.

Also note that the results of \cref{sec:pheno:bench} (i.e.\ \cref{fig:LHAbench,fig:Apfelbench_pto,fig:Pegasusbench_pto}) confirm that
the interpolation error can be kept below the benchmark accuracy.


\subsection{Matching}
\label{sec:eko/pheno-match}
\eko{} can perform calculation in a fixed flavor number scheme (\ffns{}) where
the number of active or light flavors $n_f$ is constant. This means that both
the beta function $\beta^{(n_f)}(a_s)$ and the anomalous dimension
$\bm{\gamma}^{(n_f)}(a_s)$ in \cref{eq:eko/dglap2} are constant with respect to
$n_f$.
However, this approximation is likely to fail either in the high energy region
$\muF^2 \to \infty$ for a small number of active flavors, or to fail in the low
energy region $\muF^2 \to \Lambda_{\text{QCD}}^2$ for a large number of active
flavors.

This can be overcome by using a variable flavor number scheme (\vfns{}) that
changes the number of active flavors when the scale $\muF^2$ crosses a
threshold $\mu_h^2$.
This then requires a matching procedure when changing the number of active
flavors, and for the \pdfs we find
\begin{align}
  \tilde{\mathbf{f}}^{(n_f+1)}(\mu_{F,1}^2)=&~ \tilde{\mathbf{E}}^{(n_f+1)}(\mu_{F,1}^2\leftarrow \mu_{h}^2) {\mathbf{R}^{(n_f)}} 
  \tilde{\mathbf{A}}^{(n_f)}(\mu_{h}^2) \tilde{\mathbf{E}}^{(n_f)}(\mu_{h}^2\leftarrow \mu_{F,0}^2)\nonumber\\
  &\qquad \times\tilde{\mathbf{f}}^{(n_f)}(\mu_{F,0}^2)
  \label{eq:eko/matching}
\end{align}
where the superscript refers to the number of active flavors and we split the
matching factor into two parts: the perturbative \acrfull{ome}
$\tilde{\mathbf{A}}^{(n_f)}(\mu_{h}^2)$, currently implemented at
\nnlo{}~\cite{Buza_1998}, and an algebraic rotation ${\mathbf{R}^{(n_f)}}$
acting only in the flavor space $\Fd$.

For backward evolution this matching has to be applied in the reversed order.
The inversion of the basis rotation matrices $\mathbf{R}^{(n_f)}$ is simple,
whereas this is not true for the \ome $\mathbf{\tilde A}^{(n_f)}$ especially
in case of \nnlo{} or higher order evolution.
In \eko{} we have implemented two different strategies to perform the inverse
matching: the first one is a numerical inversion, where the OMEs are inverted
exactly in Mellin space, while in the second method, called \texttt{expanded},
the matching matrices are inverted through a perturbative expansion in $a_s$,
given by:
\begin{align}
  \qty(\mathbf{\tilde A}^{(n_f)})_{exp}^{-1}(\mu_{h}^2) = \mathbf{I}
  &- a_s(\mu_{h}^2) \mathbf{\tilde A}^{(n_f),(1)} \nonumber\\
  &+ a_s^2(\mu_{h}^2) \qty[ \mathbf{\tilde A}^{(n_f),(2)} - \qty(\mathbf{\tilde A}^{(n_f),(1)})^2 ] \nonumber\\
  &+ O(a_s^3)
  \label{eq:eko/invmatchingexp}
\end{align}
with $\mathbf{I}$ the identity matrix in flavor space.


\subsection{Backward}
\label{sec:eko/pheno-back}
As a consistency check we have performed a closure test verifying that after
applying two opposite \ekos{} to a custom initial condition
we are able to recover the initial \pdf{}. Specifically, the product of the
two kernels is an identity both in flavor and momentum space up to
the numerical precision. The results are shown in \cref{fig:closure_test} in case of \nnlo{} evolution
crossing the bottom threshold scale $\mu_{F}=m_{b}$. The differences between
the two inversion methods are more visible for singlet-like quantities,
because of the non-commutativity of the matching matrix $\tilde{\mathbf{A}}_{S}^{(n_f)}$.  

\begin{figure}
    \begin{center}
    \includegraphics[width=\textwidth]{ch-eko/closure_test.pdf}
    \end{center}
    \caption{Relative distance of the product of two opposite \nnlo{} \ekos{}
        and the identity matrix, in case of exact inverse and expanded
        matching (see \cref{eq:eko/invmatchingexp}) when crossing the bottom
        threshold scale $\mu_{b}^2=\SI[parse-numbers=false]{4.92^2}{\GeV^2}$. In particular the lower scale is chosen $\muF^2=\SI[parse-numbers=false]{4.90^2}{\GeV^2}$, 
        while the upper is equal to $\muF^2=\SI[parse-numbers=false]{4.94^2}{\GeV^2}$, 
        \label{fig:closure_test}
    }
\end{figure}

Special attention must be given to the heavy quark distributions which are
always treated as intrinsic, when performing backward evolution.
In fact, if the initial \pdf{} (above the mass threshold) contains an intrinsic contribution, this has to be evolved
below the threshold otherwise momentum sum rules can be violated.
This intrinsic component is then scale independent and fully decoupled
from the evolving (light) \pdfs.
On the other hand, if the initial \pdf{} is purely perturbative, it vanishes
naturally below the mass threshold scale after having applied the
inverse matching.
In this context, \eko{} has been used in a recent study to determine, for the first time,
the intrinsic charm content of the proton~\cite{Ball:2022qks}.


\subsection{\msbar{} masses}
\label{sec:eko/pheno-msbarmass}
In the context of \pdf{} evolution, the most used treatment of heavy quarks masses are the pole masses,
where the physical values are specified as input and do not depend on any scale.
However for specific applications, such as the determination of \mhou{} due to heavy quarks contribution 
inside the proton~\cite{Ball:2016neh}, \msbar{} masses can also be used.
In particular, in~\cite{Alekhin:2010sv} it is found that higher order corrections on heavy quark production
in \dis{} are more stable upon scale variation when using the \msbar{} scheme.
\eko{} allows for this option as it is discussed briefly in the following paragraphs.

Whenever the initial condition for the mass is not given at a scale coinciding with
the mass itself (i.e. $\mu_{h,0} \neq m_{h,0}$, being $m_{h,0}$ the given initial condition
at the scale $\mu_{h,0}$),
\eko{} computes the scale at which the running mass $m_{h}(\mu_h^2)$ intersects
the identity function.
Thus for each heavy quark $h$ we solve:
%
\begin{equation}
    m_{\overline{MS},h}(m_h^2) = m_h
\end{equation}
The value $m_h(m_h)$ is then used as a reference to define the evolution thresholds.

The evolution of the \msbar{} mass is given by:
%
\begin{equation}
    m_{\overline{MS},h}(\mu_h^2) = m_{h,0} \exp\qty[ - \int\limits_{a_s(\mu_{h,0}^2)}^{a_s(\mu_h^2)} \frac{\gamma_m(a_s')}{\beta(a_s')} \dd{a_s'} ]
    \label{eq:eko/msbarsolution}
\end{equation}
%
with $\gamma_m(a_s)$ the \qcd{} anomalous mass dimension available up to 
\nnnlo{}~\cite{Vermaseren:1997fq,Schroder:2005hy,Chetyrkin:2005ia}.

Note that to solve \cref{eq:eko/msbarsolution} $a_s(\mu_R^2)$ must be evaluated in 
a \ffns{} until the threshold scales are known. Thus it is important
to start computing the \msbar{} masses of the quarks which are closer to the
the scale $\mu_{R,0}$ at which the initial reference value $a_s(\mu_{R,0}^2)$ is given. 

Furthermore, to find consistent solutions the boundary condition of the
\msbar{} masses must satisfy $m_h(\mu_h) \ge \mu_h$ for heavy quarks involving
a number of active flavors greater than the number of quark flavors $n_{f,0}$ at $\mu_{R,0}$, implying that we find
the intercept between the \rge{} and the identity in the forward direction ($m_{\overline{MS},h} \ge \mu_h$).
The opposite holds for scales related to fewer active flavors.

