% !TeX root = ../../../main.tex

An important direction for ongoing and future studies of 
new physics beyond the Standard Model (BSM) at
the Large Hadron Collider (LHC) is the search for novel heavy resonances.
%
The LHC is uniquely suited to direct searches for these resonances,
thanks to its unparalleled center of mass energy, $\sqrt{s}=13.6$ TeV in the
recently started 
Run III, and the  high statistics to be accumulated in the coming years,
especially in the high-luminosity (HL)
phase.
%
For instance, considering representative
benchmark BSM scenarios, the HL-LHC is sensitive~\cite{CidVidal:2018eel} to 
searches for sequential Standard Model (SM) 
$W'$ gauge bosons 
up to $m_{W'}=7.8$ TeV, $E_6$ model $Z'$ gauge bosons up to $m_{Z'}=5.7$ TeV, and
Kaluza-Klein resonances decaying into a $t\bar{t}$ pair up to
$m_{KK}=6.6$ TeV.

The production of such high-mass states proceeds via partonic
scattering that involves large 
values of the momentum fractions $x_1$ and $x_2$ of the colliding
partons, because the center of mass energy of the partonic collision
is $\hat s= x_1 x_2 s$.
%
For instance, the on-shell production  of a state
with invariant mass $m_{X}=8$ TeV requires
$x_1x_2 \gsim 0.3 $, hence for central production at leading order
$x_1=x_2\approx 0.6$. This is problematic because parton distribution functions
(PDFs)~\cite{Gao:2017yyd,Kovarik:2019xvh} are poorly known for $x\gsim
0.4$, as there is limited data
included in current PDF determinations to constrain this kinematic region.
%
Indeed, in the past, claims of possible BSM signals~\cite{CDF:1996yow} 
were subsequently traced to poor modeling of the PDFs in the large-$x$
region~\cite{Lai:1996mg}.
%
The impact of lack of knowledge of the PDFs
on BSM searches is thus a delicate issue~\cite{Beenakker:2015rna}.

Here we wish to further investigate this by specifically considering
neutral-current (NC) Drell-Yan (DY) dilepton production and associated
observables, frequently  used for BSM searches at the LHC.
%
NC Drell-Yan production
is one of the cleanest processes in the search for  both narrow and
broad heavy resonances decaying into dileptons,
$pp \to X \to \ell^+\ell^-$, since 
the two charged leptons can be detected with excellent energy and
angular resolution.
%
This also enables the search for
smooth, non-resonant distortions with respect to the SM
backgrounds, such as those arising in the context of contact interactions 
or, more generally,
induced by Effective Field Theory
(EFT) higher-dimensional operators that lead to 
direct couplings between quarks and leptons~\cite{Ethier:2021bye,Dawson:2018dxp,Ellis:2020unq,Greljo:2021kvv}.
Indeed, both ATLAS
and CMS have extensively explored this channel in their BSM search
program~\cite{ATLAS:2014gys,ATLAS:2020yat,ATLAS:2019erb,CMS:2021ctt,ATLAS:2021mla,CMS:2018nlk}.
To this purpose, it is mandatory to have a detailed understanding of
the dominant SM background, namely dilepton
production from quark-antiquark annihilation mediated by
a virtual electroweak (EW) boson, $q\bar{q} \to  \gamma^*/Z \to \ell^+\ell^-$,
with subleading processes involving the quark-gluon
and photon-photon initial states.

Drell-Yan production
is one of the SM processes which is known to highest perturbative
accuracy: indeed, both N$^3$LO QCD results~\cite{Duhr:2021vwj}
and the full mixed QCD-EW corrections at NNLO~\cite{Buccioni:2020cfi,Buccioni:2022kgy,Bonciani:2020tvf,Bonciani:2021zzf,Armadillo:2022bgm}
have become available recently.
%
Therefore, the main uncertainty on
theoretical predictions for this process is mostly due to the
PDFs, which, as mentioned, are poorly known at large $x$.
%
Experimentally, uncertainties are minimized when considering
observables in which several systematics cancel in part or entirely.
%
An example relevant for the DY process is the forward-backward asymmetry $A_{\rm fb}$ of the
angular distribution of the dilepton pair in the center-of-mass frame
of the partonic collision, i.e.\ the asymmetry in the
so-called Collins-Soper angle $\theta^*$, recently
measured from the Run II dataset by ATLAS~\cite{ATLAS:2017rue} and CMS~\cite{CMS:2022uul}.
%
The sensitivity of this observable to both PDFs and BSM signals has
been emphasized recently~\cite{Fiaschi:2021sin,Fiaschi:2021okg,Accomando:2019vqt,Accomando:2018nig},
as well as its relevance to extractions of the weak mixing angle
$\sin^2\theta_W$ at the LHC~\cite{CMS:2018ktx}.
These studies  are mostly restricted to the vicinity of the $Z$-boson
peak, $m_{\ell\bar{\ell}} \sim m_Z$ with $m_{\ell\bar{\ell}}$ being the dilepton mass, though in
 a recent study by CMS~\cite{CMS:2022uul} the forward-backward asymmetry has been used
to obtain a lower mass limit
(of 4.4~TeV) on a hypothetical $Z'$ heavy gauge boson.

In this work, we assess to which extent  different assumptions on the large-$x$ behavior of PDFs, as well as different
estimates of the PDF uncertainty in this region, may affect BSM searches,
by specifically studying neutral-current Drell-Yan production, and
the forward-backward asymmetry in particular. To this purpose, we 
explain the dependence of the general
qualitative features of the asymmetry   on the
behavior of PDFs, based on an understanding of the analytic dependence
of the asymmetry on the partonic luminosities.
%
We then present
detailed computations of the forward-backward asymmetry at the LHC, with realistic experimental
cuts, using a variety of PDF sets.

We find that first, the large-$x$ PDF shape and uncertainty can differ
considerably between different PDF sets, with
NNPDF4.0~\cite{Ball:2021leu} generally displaying a more flexible shape
and a wider uncertainty.
%
And second, that all PDF sets except NNPDF4.0 lead to a qualitative
behavior
of the asymmetry which in the large-mass
multi-TeV region reproduces the shape found around the $Z$-peak
region, even though there is no fundamental reason why this should be
the case
%
We will then trace the observed behavior of the asymmetry to that of the
underlying PDFs.

The structure of the paper is the following.
%
First  in Sect.~\ref{sec:HMDY} we review the leading-order (LO) expressions
for the Drell-Yan differential distributions and forward-backward asymmetry, in order to 
explain  how the leading qualitative behavior of the
asymmetry  --- specifically the reason for an asymmetry, and
its sign --- is related to the underlying parton luminosities. We 
will also show that this LO picture is not qualitatively modified 
by higher-order perturbative corrections. 
%
Then in Sect.~\ref{sec:largexpdfs} we investigate the way the
shape of the asymmetry (and specifically its sign) is determined
by the large-$x$ behavior of the PDFs. After discussing this in a 
toy model, we examine current PDF sets:
ABMP16~\cite{Alekhin:2017kpj},
CT18~\cite{Hou:2019efy}, MSHT20~\cite{Bailey:2020ooq}, and
NNPDF4.0.
%
Specifically, we  compare the
behavior of the PDFs and the asymmetry as the
final-state dilepton invariant mass is varied.
%
Finally, in Sect.~\ref{sec:afb} we present predictions 
for high-mass DY production, specifically 
the forward-backward asymmetry, at the LHC with realistic experimental
cuts, and accounting for NLO QCD and electroweak corrections.
%
For completeness, we present in App.~\ref{app:nnpdf31} a comparison to results
obtained using the previous, widely used  NNPDF3.1 PDF set.




