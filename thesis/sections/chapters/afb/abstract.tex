We discuss the sensitivity of theoretical predictions of observables used in
searches for new physics to parton
distributions (PDFs) at large momentum fraction $x$.
%
Specifically, we consider the neutral-current Drell-Yan production of
gauge bosons with invariant masses in the TeV range, for which    
the forward-backward asymmetry of charged leptons
from the decay of the gauge boson in its rest frame is a traditional
probe of new physics. We show that the qualitative  behaviour of the asymmetry 
depends strongly on the assumptions made in determining the underlying PDFs.
%
 We discuss and compare the large-$x$
 behaviour of various different PDF sets, and find that they 
 differ significantly.
 %
 Consequently, the shape of the asymmetry observed at lower
 dilepton invariant masses, where all PDF sets are in reasonable agreement 
because of  the presence of experimental constraints,
 is not necessarily reproduced at large masses where the
 PDFs are mostly unconstrained by data.
%
 It follows that the shape 
of the asymmetry at high masses may depend on 
assumptions made in the PDF parametrization, 
and thus deviations from the traditionally expected behaviour cannot be taken as a reliable 
indication of new physics.
%
We demonstrate that forward-backward asymmetry measurements
could help in constraining PDFs at large $x$ and discuss the accuracy that would be required to
disentangle the effects of new physics from uncertainties in the PDFs in this region.
