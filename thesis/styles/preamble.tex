% !TeX root = ../main.tex

% font encoding:
% NOTA BENE! richiede una distribuzione *completa* di LaTeX,
% per esempio TeXLive o MiKTeX *complete*
\usepackage[T1]{fontenc}
% \usepackage{fontspec}

% input encoding:
% [latin1] is also fine
% ACHTUNG! Sync to editor preferences
\usepackage[utf8]{inputenc}

% to write both in Italian and English;
% last language is the default one
\usepackage[italian,english]{babel}

% images
\usepackage{graphicx}

% quotes
\usepackage[font=small]{quoting}

% math
\usepackage{amsmath,amssymb,amsthm}

% complete page references
\usepackage[english]{varioref}

% fixed width tables
\usepackage{tabularx}

% double quotes optimized for biblatex
\usepackage[autostyle,italian=guillemets]{csquotes}

% bibliography;
% make citing style author-year;
% the style "numeric-comp" makes numeric references;
\usepackage[style=philosophy-modern, url=false,
			hyperref,square,backend=biber]{biblatex}

% biblatex database
\addbibresource{library.bib}

% sub-figures and sub-tables
\usepackage{subfig}

% pseudo-text
\usepackage{lipsum}

% ClassicThesis style
% - chapter numbers in Euler font
% - if using subfig package
% - Bera Mono as monospace font
% - AMS Euler as math font
% - improve line spacing
% - codes
% - for documents divided in parts 
\usepackage[eulerchapternumbers,%
            subfig,%
            beramono,%
            eulermath,%
            pdfspacing,%
            listings,%
            parts,%
            ]{classicthesis}
% rearrange the appearance of classicthesis
\usepackage{arsclassica}

\newlength{\myshift}
\setlength{\myshift}{4mm}
\addtolength{\evensidemargin}{-\myshift}
\addtolength{\oddsidemargin}{\myshift}
%\setlength{\evensidemargin}{8.40024mm} % initial value: 12.40024mm
%\setlength{\oddsidemargin}{3.80026mm} % initial value: -0.19974mm

\usepackage{datetime2}
\usepackage{datetime2-calc}
\usepackage{ifthen}
\usepackage{etoolbox}
\usepackage{calc}
\usepackage[sort]{cleveref}

% glossaries (after hyperref)
\usepackage[acronym,automake,nomain,section=section]{glossaries}
%\newglossary*{sym}{Symbols}
\newglossary*{symbols}{Symbols}
\renewcommand*{\glstextformat}[1]{\textcolor{CTurl!30!black}{#1}} %color defined by classichthesis
\makeglossaries
\newcommand{\gsym}[1]{\glssymbol{#1}}

\usepackage{ifoddpage}
% !TeX root = ../main.tex

\usepackage{microtype}

% manage layouts
\usepackage{geometry}

% indent first paragraph of each section
\usepackage{indentfirst}

% fixed width tables
% \usepackage{tabularx}

\usepackage{booktabs}
\usepackage{subfig}
\usepackage[bottom]{footmisc}
\usepackage{float}
%\usepackage[bottom]{footmisc}		% note a fondo pagina
%\usepackage{emptypage}
\usepackage{verbatim}
\usepackage{lipsum}
\usepackage{enumerate}
%\usepackage{enumitem}
\usepackage{changepage}
\let\checkmark\relax
\usepackage{dingbat}
\usepackage{lettrine}

%*********************************************************************************
% Color Boxes
%*********************************************************************************
%\usepackage{layouts}
%\printinunitsof{mm}{\pagevalues}
%\verb|\marginparwidth|: \printinunitsof{mm}\prntlen{\marginparwidth}

%*********************************************************************************
% Tikz
%*********************************************************************************
\usepackage{pgf}
\usepackage{tikz}
\usepackage{rotating}
\usepackage{pgfornament}
% \usepackage{pgfornament,pgfornament-han}
\usetikzlibrary{arrows,shapes,decorations,automata,backgrounds,petri,positioning}

\setlength{\skip\footins}{15pt}

%*********************************************************************************
% Color Boxes
%*********************************************************************************

\usepackage{tcolorbox}
\tcbuselibrary{theorems}

%*********************************************************************************
% Optimized margins for B5
%*********************************************************************************
\KOMAoptions{BCOR=10mm}
\areaset[current]{360pt}{650pt}
\setlength{\marginparwidth}{7em}
\setlength{\marginparsep}{2em}%

%*********************************************************************************
% Minitoc settings
%*********************************************************************************
\usepackage{minitoc}
\dominitoc[e]
\mtcsetrules{minitoc}{off}
\mtcsetfeature{minitoc}{before}{\textcolor{black!70}{\rule{0.8\textwidth}{0.1pt}}\vspace{-20pt}}
\mtcsetfeature{minitoc}{after}{\vspace{-20pt}\textcolor{black!70}{\rule{0.8\textwidth}{0.1pt}}}
\setcounter{tocdepth}{2}

%*********************************************************************************
% listings settings - code snippets
%*********************************************************************************
\usepackage{listings}
\lstset{language=[LaTeX]Tex,%C++,
    keywordstyle=\color{RoyalBlue},%\bfseries,
    basicstyle=\small\ttfamily,
    %identifierstyle=\color{NavyBlue},
    commentstyle=\color{Green}\ttfamily,
    stringstyle=\rmfamily,
    numbers=none,%left,%
    numberstyle=\scriptsize,%\tiny
    stepnumber=5,
    numbersep=8pt,
    showstringspaces=false,
    breaklines=true,
    frameround=ftff,
    frame=single
} 


% !TeX root = ../main.tex

\def\toinf#1{\mathrel{\mathop{\sim}\limits_{\scriptscriptstyle
{#1\rightarrow\infty }}}}


% mycommands
\newcommand{\ttmp}[1]{\textit{\textcolor{violet}{#1}}}

%\newcommand{\Section}[2][]{
%	\ifstrempty{#1}{%
%		\section{#2}
%	}{%
%		\setcounter{section}{#1-1}
%		\section{#2}
%	}%
%}
